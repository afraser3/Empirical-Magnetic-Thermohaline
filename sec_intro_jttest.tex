
Observations of globular clusters and low-metallicity field stars show significant changes in the abundance ratios of elements known to be sensitive to mixing, including \ctwelvecthirteen, lithium, and [C/N], as a star evolves up the red giant branch \citep{Carbon1982, Pilachowski1986, Kraft1994, Shetrone2019}. These changes occur around the Red Giant Branch Bump (RGBB) and are largest in the most metal--poor stars \citep[e.g.][]{Gratton2000}. Large samples of stars that can be used to trace this mixing are now available from a variety of spectroscopic surveys, including GALAH \citep{buder2019}, APOGEE \citep{DR17}, GAIA-ESO \citep{Magrini2021b}, and others. \textbf{However, observed surface abundance trends are in tension with standard theoretical stellar evolution models, which predict that the surface chemistry should not evolve in this regime}.
%and observations of low--mass red giants, which show clear changes surface chemistry as a function of time.}

%In contrast, standard theoretical models of stellar evolution predict that the surface chemistry should not evolve in this regime. %In standard models,
%\red{Meridith will identify the correct paragraph of this information to include}

As low--mass stars ascend the red giant branch, they undergo a series of mixing and homogenizing events as their interior burning and energy transport zones interact. Near the base of the red giant branch, the surface convection zone reaches its deepest level of penetration into the stellar interior, leaving behind a chemical discontinuity from which it recedes in subsequent evolution. This inflection in the convection zone's movement is known as the ``first dredge up.'' The red giant branch bump occurs when the outward--propagating hydrogen burning shell encounters this chemical discontinuity, triggering a structural realignment in which the star's core contracts and the luminosity drops, causing a disruption to the otherwise monotonic increase in luminosity along the red giant branch. In one dimensional (1D) stellar models, the sensitivity of the RGBB to physical assumptions makes it a powerful diagnostic of interior mixing processes \citep[e.g.][]{Christensen-Dalsgaard:2015, Joyce2015, Khan2018}. 
%
However, in standard models, \textbf{the post dredge--up 
%\red{JT says make clear the 'these' does not include the first dredge up} 
interior mixing events} are not reflected in the abundances at the surface of the star because there is no process by which to mix heavier elements from the interior burning regions to the surface convection zone. 
%\red{i suppose li is burned in a different place than H, but super pendantic}  %\textbf{[more citations here]}.


\textbf{The most widely studied mechanism for possibly rectifying this discrepancy is thermohaline mixing, first identified in this context by \citet{charbonnel_thermohaline_2007}.
As the hydrogen-burning shell expands into the region chemically homogenized by the recent first dredge up, the $^3$He($^3$He, 2p)$^4$He reaction creates an inversion of the mean molecular weight $\mu$. While this $\mu$ inversion is insufficient to generate a convective region, 
%it can initiate thermohaline mixing. 
these conditions give rise to the 
%\textbf{The thermodynamic and chemical conditions}, while not formally unstable to convection the mean molecular weight inversion set up by the stellar evolution gives rise to a
\textit{thermohaline instability}, a phenomenon perhaps best known in the context of salt water in Earth's oceans \citep{Stern1960,baines_gill_1969}.} 
%\textbf{The most widely studied mechanism for possibly explaining these discrepancies is thermohaline mixing, first identified in this context by \citet{charbonnel_thermohaline_2007} }
%\sout{identified one promising mechanism for possibly explaining these discrepancies: thermohaline mixing. }
%%
%
%Thermohaline mixing \textbf{is a phenomenon present in some stellar radiation zones.} \sout{is a phenomenon that drives chemical mixing in some stellar radiation zones.} 
%This instability drives \textit{thermohaline mixing}, a candidate source of extra mixing in these regions. 

Thermohaline mixing is a double-diffusive phenomenon present in fluids that have different diffusivities for heat and chemical composition which in turn make opposing contributions to the vertical density gradient \citep{Turner:1974}. Thermohaline mixing then occurs in Ledoux-stable regions whose temperature gradients are stably stratified but which have an unstable mean molecular weight stratification \citep[see][for a full review]{garaud_DDC_review_2018}. This process may then, in theory, facilitate the mixing of heavy elements from the interior of the star into the surface regions, thus producing measurable changes in the surface mixing diagnostics significantly after the first dredge--up.

Given that the physical conditions required to trigger the thermohaline instability are in place at around the same time that extra mixing has been observed in red giant stars \citep[e.g.][]{Lagarde2015}, 
%\textbf{[have we adequately established that mysterious extra mixing does occur in stars by this point? -MJ]} [In the first paragraph, right? -AF], 
most authors have assumed that all of the observed extra mixing can be attributed to the thermohaline instability \citep[e.g.][]{Kirby2016, Charbonnel2020, Magrini2021a}. However, this connection has also been questioned for a variety of reasons. 

%On the observational side, 
\textbf{First, reproducing the observed amounts of mixing in this regime with 1D models requires the adoption of much} higher efficiency parameters than most fluid simulations would suggest are reasonable or physical \citep{Denissenkov2010thermohaline, denissenkov_merryfield_2011, traxler_etal_2011, brown_etal_2013}. \sout{There have also been questions about \textbf{whether observations can shed light on} the interactions of the thermohaline instability with other stellar processes including rotation \citep{Lagarde2011}, magnetism \citep{harrington}, and
dynamical shear \citep{CantielloLanger2010}, both in combination and through direct interactions 
\citep{Maeder2013, SenguptaGaraud2018, harrington}.}
%\textbf{MC: Not exactly sure what next sentence mean}
Questions have likewise been raised about whether the location of the observed extra mixing is truly consistent with thermohaline models \citep[see e.g.][]{Angelou2015, Henkel2017, TayarJoyce22}.

%In addition to these primarily observational concerns, 
There has also been some debate about how the fluid instability should be parameterized in one dimension. \textbf{Authors have proposed a variety of different prescriptions informed by simulations \citep[e.g.~][]{traxler_etal_2011,brown_etal_2013}.} However, RGB stars have Prandtl numbers of the order $\mathrm{Pr} \sim 10^{-6}$, whereas modern fluid simulations can only probe as low as $10^{-2} - 10^{-3}$. This has generated justifiable skepticism about whether we should trust 1D formalisms that are informed by simulations only in the distinctly non-stellar $\mathrm{Pr} = 10^{-2}$ regime. \red{do we need to quote the number again?} Likewise, models of thermohaline instability that include the presence of a relatively low-amplitude magnetic field can result in much larger diffusivities \citep{harrington}. 

Given both these observational and theoretical questions, \textbf{the development of a framework through which we can determine} whether signatures from true stellar conditions (observations, $\mathrm{Pr} = 10^{-6}$) are qualitatively consistent with fluid models \textbf{is timely and imperative}. In this paper, we put forth such a framework. We demonstrate \textbf{a robust and model-agnostic means of relating the} non-dimensional fluid parameters relevant to thermohaline mixing to the observed mixing around the RGB bump and show that this correlation is \textbf{indeed} qualitatively consistent with \sout{hydrodynamical} \textbf{1D} prescriptions of thermohaline mixing \textbf{informed by 3D simulations}.

This paper is organized as follows: we begin by summarizing the formalism and stellar structure quantities relevant to thermohaline mixing (Sec.~\ref{sec:formalism}). This is followed by a description of various 1D mixing prescriptions commonly adopted in stellar evolution calculations (Sec.~\ref{sec:parameterizations}). %\sout{[think that somewhere, maybe in the previous paragraph, we need to lay out why this deserves its own section. Maybe by placing more emphasis on the issue of there being multiple conflicting prescriptions, and our aim of being able to place constraints on the prescriptions -AF].}
We then introduce a suite of 1D MESA simulations and calculate the relevant fluid parameters \textbf{in the thermohaline region}
%at the location where thermohaline mixing is expected to operate, 
across a range of masses and metallicities \sout{([Fe/H]) } (Secs.~\ref{sec:mesa_experiment} and \ref{sec:mesa_results}). 
Finally, we compare an observational proxy of extra mixing, the decrease in [C/N] near the RGBB, to theoretical trends predicted by existing 1D thermohaline mixing prescriptions (Sec.~\ref{sec:obs} and Sec.~\ref{sec:punchline}). \textbf{Our results and their implications are discussed in Sections \ref{sec:punchline} and \ref{sec:conclusions}.} 
%\textbf{MC: What about Results and Conclusions ?}
