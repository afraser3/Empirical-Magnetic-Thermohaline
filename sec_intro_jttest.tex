%
%
%
\subsection{Astrophysical Context}
Observations of globular clusters and low-metallicity field stars show significant changes in the abundance ratios of elements known to be sensitive to mixing, including \ctwelvecthirteen, lithium, and [C/N], as a star evolves up the red giant branch \citep{Carbon1982, Pilachowski1986, Kraft1994, Shetrone2019}. These changes occur around the red giant branch bump (RGBB) and are largest in the most metal-poor stars \citep[e.g.][]{Gratton2000}. Large samples of stars that can be used to trace this mixing are now available from a variety of spectroscopic surveys, including GALAH \citep{buder2019}, APOGEE \citep{DR17}, and GAIA-ESO \citep{Magrini2021b}. However, observed surface abundance trends are in tension with standard theoretical stellar evolution models, which predict that the surface chemistry should not evolve in this regime.

As low-mass stars ascend the red giant branch, they undergo a series of mixing and homogenizing events as their interior burning and energy transport zones interact. Near the base of the red giant branch, the surface convection zone reaches its deepest level of penetration into the stellar interior, leaving behind a chemical discontinuity from which it recedes in subsequent evolution. This inflection in the convection zone's movement is known as the ``first dredge-up.'' The red giant branch bump occurs when the outward-propagating hydrogen burning shell encounters this chemical discontinuity, triggering a structural realignment in which the star's core contracts and the luminosity drops, causing a disruption to the otherwise monotonic increase in luminosity along the red giant branch \citep{Christensen-Dalsgaard:2015}. In one dimensional (1D) stellar models, the sensitivity of the RGBB to physical assumptions makes it a powerful diagnostic of interior mixing processes \textbf{\citep[e.g.][]{FusiPecci1990,Salaris2002, Joyce2015, Khan2018}}. 
%
However, in standard models of red giant stars, there is no mixing between the hydrogen-burning shell and the overlying convective envelope after the first dredge-up, and no change in surface abundances is predicted in this regime. This is in direct conflict with abundance trends found in observations.

The most widely studied candidate mechanism for rectifying this discrepancy is thermohaline mixing, driven by \textbf{an inversion of} the mean molecular weight $mu$ caused by $^3$He burning. 
\textbf{The potential for $^3$He burning to cause a $\mu$ inversion and drive thermohaline mixing (though not specifically in low-mass RGB stars) was first pointed out by \citet{Ulrich1972}. Later, this $\mu$ inversion was identified as a significant source of mixing in low-mass RGB stars \citep{Dearborn2006, Eggleton2006, Eggleton2007}, but was connected to the Rayleigh-Taylor instability, not thermohaline mixing. The specific connection between extra mixing in low-mass RGB stars and \textit{thermohaline mixing} was later made by \citet{charbonnel_thermohaline_2007}}. As the hydrogen-burning shell moves into the region chemically homogenized by the  first dredge-up, the $^3$He($^3$He, 2p)$^4$He reaction creates an inversion of the mean molecular weight $\mu$. While this $\mu$ inversion is insufficient to generate a convective region (c.f. \citealt{CantielloLanger2010}), these conditions give rise to the 
\textit{thermohaline instability}, a phenomenon perhaps best known in the context of salt water in Earth's oceans \citep{Stern1960,baines_gill_1969}. 

\subsection{Fluid Dynamics Context} \label{intro:subsec:fluids}
Thermohaline mixing occurs in Ledoux-stable regions that have stably stratified temperature gradients but unstable mean molecular weight stratification 
\textbf{(see, e.g., \citealt{garaud_DDC_review_2018} for a full review or \citealt{SalarisCassisi2017} for discussion focused on a 1D stellar modeling context).}
Thermohaline mixing is a double-diffusive phenomenon present in fluids that have different diffusivities for heat and chemical composition that, in turn, make opposing contributions to the \textbf{radial} density gradient \citep{Turner:1974}.
%\citep[see][for a full review]{garaud_DDC_review_2018} 
This process may facilitate the \textbf{radial} mixing of elements between the hydrogen-burning shell and the stellar convective envelope, thus producing measurable changes in the surface mixing diagnostics after the first dredge-up. 

\textbf{Fluid dynamicists have studied thermohaline mixing in great detail, often employing 3D simulations in ``local'' domains, i.e., periodic boundary conditions with constant linear background gradients in temperature $T$ and mean molecular weight $\mu$ under the Boussinesq approximation \citep{spiegel_boussinesq_1960}. Within this standard framework, the efficiency of thermohaline mixing, $\Dth$ (the degree to which thermohaline mixing enhances chemical mixing via turbulent motions, as explained in Sec.~\ref{sec:parameterizations} below), strictly depends on three dimensionless numbers, which we introduce here but describe in further detail in Secs.~\ref{sec:formalism} and \ref{sec:parameterizations}. %The Prandtl number $\mathrm{Pr}$ and diffusivity ratio $\tau$ characterize molecular diffusivities and are independent of background gradients.
The Prandtl number $\mathrm{Pr}$ and diffusivity ratio $\tau$ characterize molecular diffusivities and are fluid properties, meaning they are independent of background gradients. %by definition.
$\mathrm{Pr}$ is the ratio of the kinematic viscosity to thermal diffusivity, and $\tau$ is the ratio of the compositional and thermal diffusivities. In stars, $\mathrm{Pr}\approx\tau \ll 1$, so double-diffusive instabilities like thermohaline mixing are readily driven \citep{garaud_DDC_review_2018}.} 

\textbf{The third dimensionless quantity describing thermohaline mixing is the density ratio $R_0$, which is the ratio of the temperature gradient's stabilizing contribution to the density divided by the $\mu$ gradient's destabilizing contribution to the density. The density ratio is a measure of how conducive to driving thermohaline mixing the the conditions in a radial location within a star are. %for driving thermohaline mixing. 
If the density ratio is large, then the destabilizing $\mu$ gradient is weak and the system may be stable (if $R_0$ exceeds a threshold that depends on $\tau$) or only weakly unstable to the thermohaline instability. A small density ratio, on the other hand, corresponds to a significant inversion of the $\mu$ profile and thus the system is strongly unstable to thermohaline mixing and possibly even convection.} 
%If the density ratio is large, then the destabilizing $\mu$ gradient is weak and the system may be stable (if $R_0$ exceeds a threshold that depends on $\tau$) or only weakly unstable to the thermohaline instability, while a small density ratio corresponds to a significant inversion of the $\mu$ profile and thus the system is strongly unstable to thermohaline mixing and possibly even convection.} 

\textbf{In this work, we study the \emph{reduced density ratio}, $r$, which combines $R_0$ and $\tau$ into a single quantity that directly determines the instability of a system to the thermohaline instability \citep{traxler_etal_2011}. The reduced density ratio directly defines the fluid dynamical stability of a system such that
\begin{equation}
r
    \begin{cases}
    \leq 0 & \mbox{System is convectively unstable} \\
    \in (0,1) & \mbox{System is thermohaline unstable} \\
    \geq 1 & \mbox{System is stable}
    \end{cases}.
\end{equation} }
\textbf{We stress that $r$ is \emph{not} a measure of the \emph{efficiency} or mixing rate of thermohaline mixing. Rather, it is a measure of the structural stability of a system, or its tendency to mix. While mixing efficiency, $\Dth$, does depend on $r$, the exact dependence is an open question. Several simplified mixing prescriptions exist for predicting how efficiency depends on different fluid parameters \citep[see review by][]{garaud_DDC_review_2018}. % and is something one might hope could be constrained by observations.}
As we will show in Sec.~\ref{sec:parameterizations}, often these mixing prescriptions diverge significantly in their predictions for how mixing efficiency ($\Dth$) varies with $r$.}

\subsection{Open Questions}
Given that the physical conditions required to trigger the thermohaline instability are in place at around the same time that extra mixing has been observed in red giant stars \citep[e.g.][]{Lagarde2015}, most authors have assumed that all of the observed extra mixing can be attributed to the thermohaline instability \citep[e.g.][]{Kirby2016, Charbonnel2020, Magrini2021a}. However, this connection has also been questioned for a number of reasons. 

First, reproducing the observed amounts of mixing in this regime with 1D models requires %\textbf{the use of thermohaline mixing prescriptions that assume much more efficient mixing} 
\textbf{assuming that thermohaline mixing is much more efficient} \textbf{than what most fluid simulations would suggest is reasonable} \citep{Denissenkov2010thermohaline, denissenkov_merryfield_2011, traxler_etal_2011, brown_etal_2013}. Questions have likewise been raised about whether the evolutionary timing of the observed extra mixing is truly consistent with thermohaline models \citep[see e.g.][]{Angelou2015, Henkel2017, TayarJoyce22}.

\textbf{Additionally, authors have put forth many different prescriptions for including thermohaline mixing in 1D models.} %, which sometimes predict significantly different trends in mixing efficiency versus fundamental fluid parameters (including $r$).} 
%and while some of them are informed by 3D fluid simulations} %\citep[e.g.~][]{traxler_etal_2011,brown_etal_2013}. 
%\textbf{Using data from observations to inform mixing prescriptions is a tempting means of addressing this concern, and identify which parameterizations are more reliable when extrapolated to $\mathrm{Pr} \sim 10^{-6}$. However, previous work has focused on using observations to constrain free parameters corresponding to overall efficiency in individual mixing prescriptions [cite cite cite], rather than to discriminate between different mixing prescriptions, which predict significantly different trends in how mixing efficiency varies with fluid parameters like $R_0$.}
\textbf{Many previous works (for example, \citealt{CharbonnelLagarde2010}, \citealt{Lagarde2017}) have used observations to calibrate or constrain 1D stellar model parameters that control the overall mixing efficiency within a particular mixing prescription (for example, $\alpha_{\text{th}}$ in the Kippenhahn prescription, described in Sec.~\ref{sec:parameterizations} below).  
These efforts generally employ a single prescription \citep[e.g.~that of][]{kippenhahn_etal_1980} and calibrate that prescription to observations. 
However, calibrating free parameters (e.g. $\alpha_{\text{th}}$, see Sec.~\ref{sec:parameterizations}) within any given mixing prescription, while useful in a 1D context, does not allow us to discriminate between the different proposed mixing prescriptions, which predict different (even, in some cases, opposite) trends in mixing efficiency $\Dth$ versus fluid parameters like $r$.}
% %, so it is unclear how the choice of prescription affects their results, and it is unclear which of the various prescriptions produces a more accurate relationship between mixing efficiency and $r$.}
\textbf{We are unaware of any instance in which observations of mixing have been used to (in)validate proposed theoretical mixing prescriptions.} 

\textbf{This issue is all the more pressing in light of recent work demonstrating that} models of thermohaline instability that include the presence of a relatively low-amplitude magnetic field can result in much \textbf{more efficient mixing (larger $\Dth$) and \emph{significantly different dependence of that efficiency on $r$}} \citep{harrington}. 
\textbf{Thus, while the issue that 1D models need to assume extremely efficient mixing in order to agree with observations could potentially be addressed by the inclusion of magnetic effects, it is not clear whether the profoundly different trend in mixing efficiency versus $r$ is consistent with observations.}

\textbf{With more observational data available in stars across different masses and metallicities, a natural question to ask is: can these data be used to identify which of the different relationships between mixing efficiency ($\Dth$) and $r$ are more consistent with observations? This question is challenging because observations show how the amount of mixing mixing varies with mass and metallicity, while different mixing prescriptions relate mixing efficiency to $r$. Fluid properties (e.g.~$\mathrm{Pr}$, $\tau$) are readily extracted from 1D stellar evolution models \citep[e.g.][]{Jermyn_Anders_atlas}, thus allowing values to be inferred for stars across different masses and metallicities. 
In contrast, we are unaware of any prior work which has extracted $r$ from 1D stellar evolution models, which is required in order to understand how observed mixing trends relate to thermohaline models developed from 3D dynamical simulations. } %Thus, when different reduced models predict different mixing efficiencies at a given $R_0$, it is not always clear whether such discrepancies are actually relevant to stellar interiors -- i.e., they might disagree at values of $R_0$ that are never achieved in stellar interiors.}

\subsection{Purpose of the present study}
%
Given these observational and theoretical questions, the development of a framework through which we can determine whether \textbf{particular thermohaline prescriptions are more or less consistent with observations} %\textcolor{red}{[``true stellar conditions" is now less meaningful now that we've shoved the $\mathrm{Pr} \sim 10^{-6}$ vs $10^{-1}$ stuff to a later section. Any suggestions?]}
is timely and imperative. 
In this paper, we put forth such a framework: one that 
\textbf{shifts the focus away from}
the calibration of individual \textbf{mixing efficiency parameters within particular 1D thermohaline mixing prescriptions,} \textbf{and instead facilitates} \textbf{inter-comparison among the mixing models themselves}. 
\begin{quote}
We demonstrate a robust and model-agnostic means of relating the non-dimensional fluid parameters relevant to thermohaline mixing \textbf{(particularly $r$)} to the observed mixing around the RGB bump and show that this correlation is indeed qualitatively consistent with 1D prescriptions of thermohaline mixing informed by 3D simulations. 
\end{quote}
Further, while previous work \citep[e.g.][]{charbonnel_thermohaline_2007} has used the measurements of the \textit{overall amount} of extra mixing to tune \textbf{model parameters that control} the \textit{overall efficiency} of thermohaline mixing prescriptions, our framework allows us to use trends in extra mixing as a function of fundamental stellar parameters to probe trends \textbf{in mixing efficiency ($\Dth$) as a function of fluid parameters}.
%\begin{quote}
\textbf{Using this framework, we demonstrate that the magnetized thermohaline mixing prescription put forth by \citet{harrington}, which addressed significant outstanding problems by enhancing mixing efficiency over hydrodynamic levels, is not consistent with observations (when fixing their magnetic parameter $H_B$).}
%\end{quote}

This paper is organized as follows: we begin by summarizing the formalism and stellar structure quantities relevant to thermohaline mixing (Sec.~\ref{sec:formalism}). This is followed by a description of various 1D mixing prescriptions commonly adopted in stellar evolution calculations (Sec.~\ref{sec:parameterizations}). We then introduce a suite of 1D MESA simulations and calculate the relevant fluid parameters in the thermohaline region for a range of mass and metallicity assumptions (Secs.~\ref{sec:mesa_experiment} and \ref{sec:mesa_results}). Finally, we compare an observational proxy of extra mixing, the decrease in [C/N] near the RGBB, to theoretical trends predicted by existing 1D thermohaline mixing prescriptions (Sec.~\ref{sec:obs} and Sec.~\ref{sec:punchline}). Our results and their implications are discussed in Sections \ref{sec:punchline} and \ref{sec:conclusions}. 