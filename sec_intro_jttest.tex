%
%
%
Observations of globular clusters and low-metallicity field stars show significant changes in the abundance ratios of elements known to be sensitive to mixing, including \ctwelvecthirteen, lithium, and [C/N], as a star evolves up the red giant branch \citep{Carbon1982, Pilachowski1986, Kraft1994, Shetrone2019}. These changes occur around the red giant branch bump (RGBB) and are largest in the most metal-poor stars \citep[e.g.][]{Gratton2000}. Large samples of stars that can be used to trace this mixing are now available from a variety of spectroscopic surveys, including GALAH \citep{buder2019}, APOGEE \citep{DR17}, and GAIA-ESO \citep{Magrini2021b}. However, observed surface abundance trends are in tension with standard theoretical stellar evolution models, which predict that the surface chemistry should not evolve in this regime.

As low-mass stars ascend the red giant branch, they undergo a series of mixing and homogenizing events as their interior burning and energy transport zones interact. Near the base of the red giant branch, the surface convection zone reaches its deepest level of penetration into the stellar interior, leaving behind a chemical discontinuity from which it recedes in subsequent evolution. This inflection in the convection zone's movement is known as the ``first dredge-up.'' The red giant branch bump occurs when the outward-propagating hydrogen burning shell encounters this chemical discontinuity, triggering a structural realignment in which the star's core contracts and the luminosity drops, causing a disruption to the otherwise monotonic increase in luminosity along the red giant branch \citep{Christensen-Dalsgaard:2015}. In one dimensional (1D) stellar models, the sensitivity of the RGBB to physical assumptions makes it a powerful diagnostic of interior mixing processes \citep[e.g.][]{Joyce2015, Khan2018}. 
%
However, in standard models of red giant stars, there is no mixing between the hydrogen-burning shell and the overlying convective envelope after the first dredge-up, and no change in surface abundances is predicted in this regime. This is in direct conflict with abundance trends found in observations.

The most widely studied candidate mechanism for rectifying this discrepancy is thermohaline mixing, identified in this context by \citet{charbonnel_thermohaline_2007} and others. As the hydrogen-burning shell moves into the region chemically homogenized by the  first dredge-up, the $^3$He($^3$He, 2p)$^4$He reaction creates an inversion of the mean molecular weight $\mu$. While this $\mu$ inversion is insufficient to generate a convective region (c.f. \citealt{CantielloLanger2010}), these conditions give rise to the 
\textit{thermohaline instability}, a phenomenon perhaps best known in the context of salt water in Earth's oceans \citep{Stern1960,baines_gill_1969}. 

Thermohaline mixing is a double-diffusive phenomenon present in fluids that have different diffusivities for heat and chemical composition which in turn make opposing contributions to the vertical density gradient \citep{Turner:1974}. Thermohaline mixing occurs in Ledoux-stable regions that have stably stratified temperature gradients but unstable mean molecular weight stratification \citep[see][for a full review]{garaud_DDC_review_2018}. This process may facilitate the vertical mixing of elements between the hydrogen-burning shell and the stellar convective envelope, thus producing measurable changes in the surface mixing diagnostics after the first dredge-up. 

Given that the physical conditions required to trigger the thermohaline instability are in place at around the same time that extra mixing has been observed in red giant stars \citep[e.g.][]{Lagarde2015}, most authors have assumed that all of the observed extra mixing can be attributed to the thermohaline instability \citep[e.g.][]{Kirby2016, Charbonnel2020, Magrini2021a}. However, this connection has also been questioned for a number of reasons. 

First, reproducing the observed amounts of mixing in this regime with 1D models requires the adoption of much higher efficiency parameters than most fluid simulations would suggest are reasonable or physical \citep{Denissenkov2010thermohaline, denissenkov_merryfield_2011, traxler_etal_2011, brown_etal_2013}. Questions have likewise been raised about whether the evolutionary timing of the observed extra mixing is truly consistent with thermohaline models \citep[see e.g.][]{Angelou2015, Henkel2017, TayarJoyce22}.

There has also been some debate about how the fluid instability should be parameterized in one dimension, and authors have proposed a variety of different prescriptions informed by numerical simulations \citep[e.g.~][]{traxler_etal_2011,brown_etal_2013}. However, RGB stars have much lower ratios of kinematic viscosity to thermal diffusivity than simulations can reach, of the order $\mathrm{Pr} \sim 10^{-6}$, whereas modern fluid simulations can only probe as low as $10^{-2} - 10^{-3}$. This has generated skepticism about whether trends from simulations can be accurately extrapolated into stellar regimes. Likewise, models of thermohaline instability that include the presence of a relatively low-amplitude magnetic field can result in much larger diffusivities \citep{harrington}, raising the question of whether earlier prescriptions that neglect magnetic fields may be missing key physics. 



Given both these observational and theoretical questions, the development of a framework through which we can determine whether signatures from true stellar conditions (observations, $\mathrm{Pr} = 10^{-6}$) are qualitatively consistent with fluid models is timely and imperative. 
In this paper, we put forth such a framework: one that allows not only the calibration of individual mixing parameterizations, but also comparison between mixing models. We demonstrate a robust and model-agnostic means of relating the non-dimensional fluid parameters relevant to thermohaline mixing to the observed mixing around the RGB bump and show that this correlation is indeed qualitatively consistent with 1D prescriptions of thermohaline mixing informed by 3D simulations. Further, while previous work \citep[e.g.][]{charbonnel_thermohaline_2007} has used the measurements of the \textit{overall amount} of extra mixing to tune the \textit{overall efficiency} of thermohaline mixing prescriptions, our framework allows us to use trends in extra mixing as a function of fundamental stellar parameters to probe trends predicted by various prescriptions.

This paper is organized as follows: we begin by summarizing the formalism and stellar structure quantities relevant to thermohaline mixing (Sec.~\ref{sec:formalism}). This is followed by a description of various 1D mixing prescriptions commonly adopted in stellar evolution calculations (Sec.~\ref{sec:parameterizations}). We then introduce a suite of 1D MESA simulations and calculate the relevant fluid parameters in the thermohaline region for a range of mass and metallicity assumptions (Secs.~\ref{sec:mesa_experiment} and \ref{sec:mesa_results}). Finally, we compare an observational proxy of extra mixing, the decrease in [C/N] near the RGBB, to theoretical trends predicted by existing 1D thermohaline mixing prescriptions (Sec.~\ref{sec:obs} and Sec.~\ref{sec:punchline}). Our results and their implications are discussed in Sections \ref{sec:punchline} and \ref{sec:conclusions}. 