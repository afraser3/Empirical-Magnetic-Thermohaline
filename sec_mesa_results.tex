%
%
Figure \ref{fig:mesa_r_spread} compares results from four physical configurations describing thermohaline mixing in MESA: the upper left panel shows results from the Brown model; the remaining three show results from the Kippenhahn prescription with $\alpha_{\text{th}}$ varying as indicated. The reduced density ratio $\log_{10} r$ is shown as a function of mass and metallicity and indicated on the color bar and grid labels.
\sout{Simulations for which our algorithm could not measure a stable thermohaline zone are blocked out in grey and have no label; this happens at high mass and low metallicity.} \red{I think all models will have thermohaline with the corrected algorithms; TBD - EA}.
%

In all cases, the most notable trend is that $\log_{10} r$ decreases along the diagonal from high masses and metallicities (upper left) to low masses and metallicities (lower right). There is particularly high qualitative similarity between the Brown model and Kippenhahn model with $\alpha_{\text{th}} = 2$, which correspond to similar thermohaline mixing timescales. The case with the lowest mixing parameterization is the Kippenhahn $\alpha_{\text{th}} = 0.1$ case, and there the span of $\log_{10} r$ values is smallest. We also note that, unlike in the other three cases, $\log_{10} r$ does not scale precisely monotonically with either mass or [Fe/H] in the Kippenhahn $\alpha_{\text{th}} = 700$ case. While there is no clear relationship between the spread of $\log_{10} r$ values observed when using the Kippenhahn prescriptions and the values of $\alpha_{\text{th}}$ adopted in each, there is a clear relationship between the median values of $\log_{10} r$ and $\alpha_{\text{th}}$: the reduced density ratios are larger  when mixing is highly efficient (i.e. $t_{\mathrm th} << t_{\text{evol}}$). 
Most importantly, the overall behavior of $\log_{10} r$ as a function of mass and [Fe/H] is consistent regardless of the theoretical assumption adopted.
%
%The overall trends of $\log_{10} r$ vs.~[Fe/H] and mass are  
%regardless of the
This robustness across 1D thermohaline mixing model assumptions suggests that $r$ may be useful as a mixing diagnostic in physical data sets. We explore its application to observations subsequently.


\begin{figure*}
    \centering
    \includegraphics[width=\textwidth]{mesa_r_spread.pdf}
    \caption{The reduced density ratio $\log_{10} r$ is extracted as discussed in Section \ref{sec:mesa_experiment} for four grids of stellar models with differing prescriptions for thermohaline mixing. 
    Results for $\log_{10} r$ are shown as a function of stellar mass and metallicity [Fe/H], with high values of $\log_{10} r$ in brighter colors (yellow) and low values of $\log_{10} r$ in darker colors (purple). 
    The model name and mixing efficiency, $\alpha_{\text{th}}$ (where applicable) constitute the physical configuration and are indicated in the panel labels.}
    \label{fig:mesa_r_spread}
\end{figure*}