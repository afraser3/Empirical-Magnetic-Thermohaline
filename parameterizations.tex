\textbf{TODO:} Clean up text and symbols throughout. Make everything a Nu or a D consistently.

The density ratio is defined \citep{ulrich_1972}
\begin{equation}
    R_\rho = \frac{\nabla_T - \nabla_{\rm{ad}}}{{\nabla_\mu}},
\end{equation}
where $\nabla_T$ is the logarithmic temperature gradient, $\nabla_{\rm{ad}}$ is its adiabatic value, and $\nabla_\mu$ is the Ledoux composition gradient term.

The default thermohaline mixing model in the MESA software instrument \citep{mesa2} was originally derived by \citet{ulrich_1972, kippenhahn_etal_1980}.
Thermohaline mixing is treated as a diffusion process, with a diffusion coefficient
\begin{equation}
    D_{\rm{th}} = \alpha_{\rm{th}} \frac{3 K}{2\rho C_P}R_{\rho}^{-1}.
    \label{eqn:kipp_model}
\end{equation}
Here, $\alpha_{\rm{th}}$ is a dimensionless efficiency parameter, $K$ is the radiative conductivity, $\rho$ is the density, and $C_P$ is the specific heat at constant pressure. 

\citet{traxler_etal_2011} derive a mixing law by fitting a function to a large set of simulations.
They define the reduce density ratio, $r \equiv (R_0 - 1)/(\tau^{-1} - 1)$, where $\tau = \kappa_\mu/\kappa_T$ is the ratio of compositional to thermal diffusivity.
They find that
\begin{equation}
    \mathrm{Nu}_{\rm{th}} - 1 = \sqrt{\frac{\mathrm{Pr}}{\tau}}\left(a e^{-br}[1 - r]^c\right),
    \label{eqn:trax_model}
\end{equation}
where $\mathrm{Pr} = \nu / \kappa_T$ with $\nu$ the viscosity, and they find $a = 101 \pm 1$, $b = 3.6 \pm 0.3$, and $c = 1.1 \pm 0.1$.
The turbulent compositional diffusivity is $D_{\rm{th}} = (\mathrm{Nu}_{\rm{th}} - 1)\kappa_\mu$.


\citet{brown_etal_2013} note that the model in Eqn.~\ref{eqn:trax_model} performs well at high $R_\rho$, but underestimates mixing at low $R_\rho$, particularly in the stellar regime of low Pr and $\tau$.
They develop a semi-analytical model for thermohaline mixing,
\begin{equation}
    \mathrm{Nu}_{\rm{th}} - 1 = C^2\frac{\lambda^2}{\tau l^2(\lambda + \tau l^2)},
\end{equation}
where $\lambda$ is the growth rate of the fastest growing linearly unstable mode, $l$ is its horizontal wavenumber, and $C \approx 7$ is a universal constant which they fit to simulation data.
Obtaining $\lambda$ and $l$ requires solving a cubic and quadratic equation (their Eqns.~19 and 20).

These new implementations are used in \citep{bauer_bildsten_2019} and other works.

(cite) paper is the paper that implements these new models in MESA and it's been there since r(put revision here)

\citet{lattanzio_etal_2015} tested one or multiple of these models in a bunch of different codes on the RGB and found X.

There has been other work to do multi-D models of thermohaline mixing \citep{denissenkov_2010, denissenkov_merryfield_2011}, but 2D thermohaline behaves very differently from 3D thermohaline \citep{garaud_brummell_2015}, and so we do not consider that set of data in this work.

\textbf{Adrian}: Start making this plot
Do you need to discuss here? or cite previous work? include plot of Nu vs R0 with the different model/theory predictions, mark simulations, like you did for bring a plot [AF: assuming I finally wrap up my in-prep paper with Pascale, we can just cite that paper and throw in those Nu vs R0 plots]