The instability driving thermohaline mixing requires a Ledoux-stable inversion of the mean molecular weight $\mu$ in the presence of a stable temperature gradient. 
Expressed in terms of typical stellar structure variables, the stability of the temperature gradient is given by the Schwarzschild criterion:
\begin{equation} \label{eq:Schwarzschild}
    \gradrad - \gradad < 0,
\end{equation}
\textcolor{red}{[Evan: please paraphrase this sentence]} where the temperature gradient $\grad \equiv d \ln P / d \ln T$ (pressure $P$ and temperature $T$) is $\gradad$ for an adiabatic stratification and $\gradrad$ if all the flux is carried radiatively. 
The statement that the temperature gradient is strong enough to ensure stability by the Ledoux criterion (i.e.~the fluid is stably-stratified) despite the inversion of the mean molecular weight is
\begin{equation} \label{eq:Ledoux}
    \gradrad - \gradad - \frac{\phi}{\delta}\gradmu < 0.
\end{equation}
\textcolor{red}{[Evan: please paraphrase this sentence]} The Ledoux criterion includes the effects of the composition gradient $\gradmu = d\ln\mu/d\ln P$ (mean molecular weight $\mu$), where $\delta = -(\partial \ln \rho / \partial \ln T)_{P,\mu}$ and $\phi = (\partial \ln \rho / \partial \ln\mu)_{P,T}$ (density $\rho$).

The stabilizing influence of the temperature gradient relative to the destabilizing influence of the $\mu$ gradient is often given in terms of the density ratio $R_0$, defined as
\begin{equation} \label{eq:R0}
    R_0 \equiv \frac{\grad - \gradad}{\frac{\phi}{\delta} \gradmu},
\end{equation}
where $R_0 < 1$ implies the $\mu$ gradient is sufficiently strong to drive convection, and $R_0 > 1$ implies the fluid is stably-stratified (i.e.~no convection). 
As reviewed by \citet{garaud_DDC_review_2018}, fluids with $R_0 > 1$ can be prone to double-diffusive instabilities whenever the thermal diffusivity $\kappa_T$ is greater than the compositional diffusivity $\kappa_\mu$. Specifically, the instability driving thermohaline mixing acts whenever
\begin{equation} \label{eq:R0_condition}
1 < R_0 < 1/\tau,
\end{equation}
\citep{baines_gill_1969} where
\begin{equation} \label{eq:tau}
    \tau \equiv \kappa_\mu/\kappa_T.
\end{equation}
Note that typical values of $\tau$ in stellar radiation zones are $10^{-6}$ or smaller, meaning even very slight inversions of the $\mu$ profile can drive thermohaline mixing, even when the temperature gradient is far from being unstable by the Schwarzschild criterion.