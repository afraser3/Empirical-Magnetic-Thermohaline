% \textbf{[Two sentences on what we're about to explain and why (i.e. the importance of establishing a uniform language); want to establish physically when thermohaline is going to kick in, and what parameters do we need to measure in stellar evolution codes to establish whether thermohaline mixing is happening]}

The following section presents a derivation of the fluid parameter known as the ``reduced density ratio,'' which we use to characterize the amount of mixing expected in thermohaline regions.  
%\sout{of various model stars}. 
%\sout{This is provided for readers who may be unacquainted with this quantity and/or its relationship to more broadly known thermodynamic parameters.} 
Section \ref{sec:parameterizations} goes into a similar level of detail regarding different prescriptions for thermohaline mixing in 1D stellar models 
to the end of summarizing the prescriptions currently implemented in MESA, and demonstrating key differences between these prescriptions.
%\textbf{to the end of making explicit the relationships between the conventions adopted in each \textbf{formalism} \blu{and \sout{clarifying} how \sout{they} the quantities in each can be computed \sout{for a particular star of interest}}.} 
%\textcolor{red}{[AF wants to change the last sentence to: ``... to the end of summarizing the prescriptions currently implemented in MESA, and demonstrating key differences between these prescriptions.]}
%Readers who are familiar with these may proceed directly to Section \ref{sec:mesa_experiment}
%[MJ added; please ask before modification]} \textcolor{red}{[maybe sleep deprivation is just making me dumb, but: what's the ``each" in the last sentence here? -AF]


%
The instability driving thermohaline mixing requires a Ledoux-stable inversion of the mean molecular weight $\mu$ stratification in the presence of a stable temperature gradient. 
The stability of the temperature gradient is given by the Schwarzschild criterion:

\begin{equation} \label{eq:Schwarzschild}
    \gradrad - \gradad < 0,
\end{equation}
where the temperature gradient $\grad \equiv d \ln P / d \ln T$ (pressure $P$ and temperature $T$) has an adiabatic value $\grad = \gradad$ and saturates to $\grad = \gradrad$ in hydrostatically stable regions where the flux is carried radiatively. 
%\sout{[We should cite Schwarzschild here but I tried to read his collected works outloud in German to Jamie and got laughed at]} \textbf{[tbh it looks to me like the overwhelming standard is to NOT cite Schwarzschild, I vote we don't worry about it. Of the 7 or so papers I have open right now, none of them cite Schwarzschild]}. 
The Ledoux criterion for convective stability is \citep{Ledoux1947}

\begin{equation} \label{eq:Ledoux}
    \gradrad - \gradad - \frac{\phi}{\delta}\gradmu < 0,
\end{equation}
which must be satisfied despite the inversion of the mean molecular weight.
The Ledoux criterion accounts for the composition gradient $\gradmu = d\ln\mu/d\ln P$, where $\delta = -(\partial \ln \rho / \partial \ln T)_{P,\mu}$ and $\phi = (\partial \ln \rho / \partial \ln\mu)_{P,T}$ (where $\rho$ is density).

The stabilizing influence of the temperature gradient relative to the destabilizing influence of the $\mu$ gradient is measured by the density ratio,

\begin{equation} \label{eq:R0}
    R_0 \equiv \frac{\grad - \gradad}{\frac{\phi}{\delta} \gradmu},
\end{equation}
where $R_0 < 1$ implies the $\mu$ gradient is sufficiently unstable to drive convection, and $R_0 > 1$ implies the fluid is stably-stratified (i.e.~no convection)\footnote{Note that $R_0>1$ is equivalent to the Ledoux criterion Eq.~\eqref{eq:Ledoux} only if $\grad = \gradrad$. Thermohaline mixing primarily mixes chemicals, but does produce some minimal thermal mixing \citep[see, e.g., Fig.~4 of][]{brown_etal_2013}; thus, $\grad \neq \gradrad$. This thermal mixing is often ignored in mixing prescriptions in 1D stellar evolution programs, however.}. 
As reviewed by \citet{garaud_DDC_review_2018}, fluids with $R_0 > 1$ can be prone to double-diffusive instabilities whenever the thermal diffusivity, $\kappa_T$, is greater than the compositional diffusivity, $\kappa_\mu$. Specifically, the instability driving thermohaline mixing acts whenever

\begin{equation} \label{eq:R0_condition}
1 < R_0 < 1/\tau,
\end{equation}
\citep{baines_gill_1969} where
\begin{equation} \label{eq:tau}
    \tau \equiv \kappa_\mu/\kappa_T.
\end{equation}
Note that in stellar radiation zones, typically $\tau \lesssim 10^{-6}$. This means that very slight inversions of $\mu$ (large $R_0$) can drive thermohaline mixing, even when the temperature gradient is strongly stable according to the Schwarzschild criterion. 

Throughout this paper, we express the density ratio $R_0$ in terms of the \textit{reduced density ratio}

\begin{equation} \label{eq:r}
    r \equiv \frac{R_0 - 1}{\tau^{-1} - 1}
\end{equation}
per, e.g., \citet{traxler_etal_2011,brown_etal_2013}.
In terms of $r$, the condition for thermohaline instability, Eq.~\eqref{eq:R0_condition}, is

\begin{equation} \label{eq:r_condition}
    0 < r < 1,
\end{equation}
where $r \leq 0$ is the threshold for convection and $r \geq 1$ corresponds to scenarios where the $\mu$ inversion is too weak to drive the thermohaline instability.