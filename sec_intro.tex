Standard models of stellar evolution predict that once the envelope of low-mass stars has reached its deepest extent on the red giant branch and homogenized the surface convection zone, the so called `first dredge up', the surface chemistry of that giant should remain relatively constant through the rest of the shell hydrogen burning phase. In contrast, observations of globular cluster (citecite) as well as low-metallicity field stars \citep{Gratton2000} noticed significant changes in the abundance ratios of elements known to be sensitive to mixing, including \c12c13, lithium, and [C/N]. These changes seemed to happen only above the red giant branch bump, where the hydrogen burning shell reaches the region where a mean molecular weight gradient was left behind by the deepest evolution of the surface convection zone. These mixing related changes seemed to be largest in the lowest metallicity stars (citecite) and many mechanisms were hard pressed to explain them (cite a bunch of things or maybe a review?). 

\textbf{Adrian:} (enter the theory of thermohaline mixing. mumble mumble ocean. describe how it works. should happen in stars. Helium 3 stuff. maybe not enough citation. maybe magnets? scaling weird. Fraser et al. simulation stuff.  suggest maybe not as efficient? But size and even slope direction of trend unclear analytically/ from simulations. 

One promising mechanism was identified by \citet{charbonnel_thermohaline_2007}: fingering convection, also known as thermohaline mixing. As the hydrogen-burning shell expands into the region that the recent dredge up has rendered chemically homogeneous, the $^3$He($^3$He, 2p)$^4$He reaction creates an inversion of the mean molecular weight $\mu$. While this inverse $\mu$ gradient is not enough to generate a Ledoux-unstable region, it does drive fingering convection. \textcolor{gray}{[OK here's an outline of some stuff that could be said here, but it's probably too much: Briefly summarize instability mechanism but refer readers to Garaud 2018 review. Then talk about Ulrich's model and Kippenhahn's model, which are the most widely used in stellar evolution models but can't be right, especially at large $R_0$ where they predict nontrivial mixing despite the instability shutting off. Mention Denissenkov's fix for this, then the successful Brown 2013 model, both of which have the very physically-reasonable implication that mixing $\to 0$ at large $R_0$. One consequence of these improved models is that thermohaline mixing appears totally insufficient for explaining observations. Bummer. Then \citet{harrington} (hereafter HG19) added magnetic fields to simulations at $R_0 = 1.45$ and $\mathrm{Pm} = 1$ and found dramatic increases in mixing. This is exciting because RGB stars only need magnetic fields on the order of a hundred Gauss to explain observations. Also exciting because (maybe only include this part if I end up publishing this in my in-prep paper) Harrington's model implies these $\mu$ gradients form fully convective layers for certain magnetic field strengths, which can have observational consequences in the AGB stage. Sadly, Fraser \& Garaud have shown that the HG19 model significantly over-predicts mixing when compared to simulations for $\mathrm{Pm} < 1$ -- a ubiquitous feature of these plasmas -- and \textit{especially so for large $R_0$}. While simulations show mixing decreases with decreased $\mathrm{Pm}$ or increased $R_0$, the HG19 model (which does not included $\mathrm{Pm}$ as a parameter but it is readily added) predicts mixing is essentially unaffected by decreased $\mathrm{Pm}$ and can stay constant or even increase as $R_0$ increases. (Maybe also say that it predicts mixing $\to \infty$ as $B_0 \to \infty$ which is unphysical.) However, by necessity due to limitations of computing resources, these simulations only explored $\mathrm{Pr} \sim 10^{-1}$, whereas these regions of RGB stars feature $\mathrm{Pr} \sim 10^{-6}$. Thus, it is unclear if the disagreements between the HG19 model and simulations, especially at large $R_0$, are due to failings of the HG19 model, or due to aspects of thermohaline mixing at $\mathrm{Pr} \sim 10^{-1}$, and thus perhaps the realistic scenario does in fact feature these strange trends in mixing vs $R_0$, but you can only see those trends at really really low $\mathrm{Pr}$. Let's see if we can get an idea via observational constraints!]}

\textbf{Adrian/Meridith:} Models predict efficiency should scale with R0 (define) (explain why). This can be calculated in 1D simulations (cite Matteo?). in order to compare to theoretical sims. it depends on both He3 abundance and (the other thing). 

In this paper, we therefore show the possible/proposed dependencies of magnetic thermohaline mixing as a function of R0 in simulations. We use one dimensional stellar evolution models to estimate the range of R0s present in real stars as a function of stellar mass and metallicity, and we use measurements of the amount of mixing in these stars from observations to compare to the predictions of the simulations. 