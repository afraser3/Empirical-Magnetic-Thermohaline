[Begin outline]

\begin{itemize}
    \item In RGB stars there is a first dredge up and then a luminosity bump
    \item Standard models predict no change in surface chemistry after the first dredge up
    \item But observations show extra mixing after the luminosity bump! Explain that, science! [Briefly summarize which observations, which stars, which log g, etc.]
    \item \citet{charbonnel_thermohaline_2007} pointed out it could be thermohaline mixing, this dumb thing that is well-studied in the oceanographic context and had been considered in various other astrophysical settings (see [cite Garaud 2018])
    \item Go into explanation of thermohaline mixing (the stuff that I had previously been writing in the first half of old-section-3, except more of the literature review stuff and less of the mathy/formalism stuff):
    \begin{itemize}
        \item Thermohaline mixing is the mixing of chemicals (heat too but it's a comparably negligible amount) by small-scale turbulent motions that are driven by a double-diffusive instability
        \item Requires a Schwarzschild-stable temperature gradient and an inversion of the $\mu$ profile that is weak enough to still be Ledoux-stable
        \item Also requires thermal diffusion to be larger than chemical diffusion, but this is always the case in stars -- the degree of discrepancy between these diffusion coefficients sets how minimal of a $\mu$ inversion is necessary to trigger thermohaline instability, and in stars this discrepancy is so vast that just about any $\mu$ inversion is immediately unstable
        
        \item The criteria for instability are easy to express in terms of stellar structure variables (see Sec.~\ref{sec:formalism}), making it pretty straightforward to use 1D stellar evolution models to identify where thermohaline mixing could plausibly occur in stars across the HR diagram [list examples]
        
        Because the quantities grad-ad, grad-rad are known at every radial position in a 1D stellar structure model, it is possible to know <values of quantities that are functions of these quantities ($R0$)> at every position. 
        
        \item 
        %But it's much harder to predict what thermohaline mixing actually does to these stars, because 
        historically it's been much more challenging to estimate the efficiency of thermohaline mixing -- just because thermohaline mixing happens doesn't mean it is efficient enough to matter, where matter means?? manifests in surface-observable ways?
        
        \item Summarize efforts to get at efficiency of thermohaline mixing in stellar interiors via analytical theory and multi-dimensional simulations
    \end{itemize}

    \item Arrive at current state of things: a broad range of mixing prescriptions available to MESA users, with 
    %no real clarity
    ongoing literature debate
    as to which model to use, and what values to set for the free parameters
    \item To make matters worse, HG19 shows that magnetic fields of very moderate strengths (plausibly present in majority of these stars) give massively different mixing efficiencies than predictions from hydro theory/3D sims
    \item APOGEE data includes a boat-load of RGB observations -- can we use them to rule out or constrain various thermohaline mixing prescriptions?
    \item Similarly: can observations be used to identify which regions of parameter space simulations should aim to clarify mixing efficiency? (I.e., if two mixing prescriptions agree everywhere except for a certain corner of parameter space, can we use observations to identify whether any stars even live in that parameter space and so whether we should even care?)
    \item In this paper, we present a framework that combines observations and theory to probe the fluid parameters that actually happen in stars
\end{itemize}

[End outline]

\textbf{As low-mass stars ascend the red giant branch, they undergo a series of mixing and homogenizing events as their interior burning and energy transport zones interact. Near the base of the red giant branch, the surface convection zone (e.g. convective envelope) reaches its deepest level of penetration into the stellar interior, leaving behind a chemical discontinuity from which it recedes in subsequent evolution. This inflection in the convection zone's movement is known as the ``first dredge up.'' As the star evolves, the outward-moving hydrogen burning shell soon encounters this chemical discontinuity. This interaction triggers a structural realignment in which the star's core contracts and the luminosity drops, causing a disruption to the otherwise monotonic increase in luminosity along the red giant branch. As the star ``hovers'' in this luminosity bin for a longer period of evolutionary time relative to the rest of the branch, this feature manifests as an over-density of stars identified as the red giant branch bump (RGBB) in stellar populations. In 1D stellar models, the sensitivity of the RGBB to physical assumptions, particularly the convective mixing length and overshooting parameters, makes it a powerful mixing diagnostic.}

Standard models of stellar evolution predict that once a low-mass star reaches the first dredge-up,
%convective envelope of a low-mass star 
%\textcolor{red}{[can we be specific about the mass range?]} on the red giant branch has reached its deepest extent, the so called `first dredge up', 
the surface chemistry of that giant should remain relatively constant through the rest of the shell hydrogen burning phase. 
In contrast, observations of globular cluster \textcolor{red}{[cite?]} as well as low-metallicity field stars \citep{Gratton2000} find significant changes in the abundance ratios of elements known to be sensitive to mixing, including \ctwelvecthirteen, lithium, and [C/N]. 
These changes seemed to happen only above the red giant branch bump
%, where the hydrogen burning shell reaches the region where a mean molecular weight gradient was left behind by the deepest evolution of the surface convection zone.
These mixing related changes seemed to be largest in the lowest metallicity stars (citecite) and many mechanisms were hard pressed to explain them (cite a bunch of things or maybe a review?). 

\textcolor{red}{[Wednesday morning Adrian beginning to write new material from here on]}

\citet{charbonnel_thermohaline_2007} identified one promising mechanism for possibly explaining these discrepancies: thermohaline mixing. As the hydrogen-burning shell expands into the region chemically homogenized by the recent dredge up, the $^3$He($^3$He, 2p)$^4$He reaction creates an inversion of the mean molecular weight $\mu$. While this $\mu$ inversion is insufficient to generate a Ledoux-unstable convective region (i.e.~the fluid remains stably-stratified), it can drive thermohaline mixing. 

Thermohaline mixing is a phenomenon that drives chemical mixing in some stellar radiation zones. 
It is a form of double-diffusive convection, i.e., mixing due to small-scale, turbulent motions in regions where disparate molecular diffusion coefficients for heat and chemicals render background temperature and composition gradients linearly unstable despite being stable according to the Ledoux criterion. 
For a comprehensive review of two forms of double-diffusive convection relevant to stellar and planetary interiors (including the mixing phenomenon sometimes referred to as ``semiconvection", which is distinct from thermohaline mixing), the reader is referred to \citet{garaud_DDC_review_2018}. 

This mixing occurs in regions that have Schwarzschild-stable temperature gradients and inversions in the mean molecular weight that are sufficiently weak to maintain stability according to the Ledoux criterion (i.e.~the fluid remains stably-stratified), but stronger than some minimum threshold determined by the ratio of molecular diffusion coefficients for heat and chemicals \textcolor{red}{[Evan: please help me be careful about ``chemical composition" vs ``mean molecular weight" just like in your paper]}. 
In stellar interiors, these diffusivities are different by many orders of magnitude, which renders even extremely slight inversions of the mean molecular weight linearly unstable to thermohaline instability \textcolor{red}{[how do we feel about saying ``thermohaline instability"]}.

The necessary and sufficient conditions for thermohaline instability are simply specified using gradients and molecular diffusivities (see Sec.~\ref{sec:formalism}), which are known at every radial position in a 1D stellar structure model. 
Therefore, it is straightforward to use 1D models to assess which stars across the HR diagram might have thermohaline mixing. [go over the different stars and recent papers, e.g., Ylva, Evan Bauer, etc. I think there's a planetary in-fall paper by Melinda too?]

While these instability criteria are readily assessed, they do not provide the rate at which chemicals are mixed by thermohaline mixing. 
This presents a challenge when determining if surface observations can be attributed to thermohaline mixing: while the occurrence of thermohaline mixing in 1D stellar evolution models is coincident with stars where extra mixing is inferred, one cannot rule out the possibility that thermohaline mixing is too inefficient to explain the inferred extra mixing. 
[Following is a dumb sentence that can probably be deleted? idk] Instability criteria and 1D stellar evolution models can be used to establish correlation between extra mixing and thermohaline mixing, but establishing causation requires more systematic study.

This has driven several efforts dating back to the 1970's to derive predictive models of thermohaline mixing, i.e., prescriptions for the efficiency at which chemicals are mixed as a function of molecular diffusivities and background gradients that can be implemented in 1D stellar evolution models. 
Much progress has been made on this front in recent years \citep[see review by][and Sec.~\ref{sec:parameterizations} of this paper]{garaud_DDC_review_2018}, leading to a range of mixing prescriptions that are implemented in 1D stellar evolution models. 
However, many of these prescriptions yield conflicting predictions for efficiency of chemical mixing. 
While the more recent of these prescriptions \citep[e.g.~][]{traxler_etal_2011,brown_etal_2013} have been validated against 3D hydrodynamic simulations, these prescriptions predict orders of magnitude less chemical mixing than the models used by \citet{charbonnel_thermohaline_2007} to explain data by \citet{Gratton2000} of RGB surface chemical abundance observations \citep[see Sec.~3.1 of][]{traxler_etal_2011}. 
Furthermore, \citet{harrington} have shown that remarkably weak magnetic fields (e.g.~of order $100$G in RGB stars, which might plausibly exist in the majority of these stars) can dramatically enhance thermohaline mixing efficiency beyond levels predicted by hydrodynamic simulations, and produce profoundly different trends for mixing efficiency versus background density gradients. 

Given the broad range of prescriptions used throughout the recent literature [and maybe mention issue of $\mathrm{Pr} \sim 10^{-1}$ vs $10^{-6}$], it therefore becomes necessary to compare them to observations to narrow things down. 
Here, the massive amount of data provided by APOGEE becomes useful [SOS Jamie help please] because of the large number of accurate and precise measurements of M thanks to asteroseismology. 
However, comparing observations to theory models requires a framework in which we can interpret the observations (e.g.~masses and surface chemical abundances) in the parameters that the theory requires (e.g.~background gradients and molecular diffusivities). 
We therefore need a large sample of observations with well-constrained estimates of mass and composition as well as a set of theoretical models which can provide the necessary fluid parameters for stars across this regime. 

Therefore, in this paper we provide: a summary of the formalism and stellar structure variables/fluid parameters relevant to thermohaline mixing (Sec.~\ref{sec:formalism}) followed by a reasonable sampling of the range of theoretical models currently used for thermohaline mixing efficiency (Sec.~\ref{sec:parameterizations}); a suite of 1D simulations to calculate the relevant fluid parameters across a range of $M$ and $Z$ (Secs.~\ref{sec:mesa_experiment} and \ref{sec:mesa_results}); and a set of measurements of the decrease in [C/N] near the RGB bump (Sec.~\ref{sec:obs}) in order to qualitatively compare the theoretical models to the observed estimates of extra mixing and determine whether the thermohaline models that exist are a good match to the observed mixing signatures.