Thermohaline mixing has long been though to be a good candidate to explain the evolution of the surface chemistry of low-mass upper red giant branch stars. We show here that

\begin{itemize}
    \item mixing rates should strongly depend on the balance of (R0- redefine)
    
    \item stellar evolution models suggest that R0 should vary as a function of mass and metallicity. The real range covered in low-mass upper RGB stars is (a lot/ a little) compared to the range of simulations that have been done. These stellar evolution models suggest that R0 should depend strongly on composition and only weakly on stellar mass
    
    \item observations suggest that the amound of mixing is/is not strongly correlated with R0 as predicted in 3D. Observations indicate that as R0 increases, the mixing rate (increases/decreases) consistent with (cite simulations) but not (cite other simulations)
    
    \item this suggests that (things about 3D models)
    
    \item magnetized thermohaline is/is not a good predictive theory for mixing in low-mass upper RGB stars
\end{itemize}
    
    If this is the case, should use this theory to predict cool stuff for lithium rich star production, mass transfer systems, etc. 

