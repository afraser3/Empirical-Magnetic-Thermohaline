Thermohaline mixing has long been though to be a good candidate to explain the evolution of the surface chemistry of low-mass upper red giant branch stars. We show here that:

\begin{itemize}
    \item three-dimensional simulations of the thermohaline instability suggest that mixing rates should strongly depend on the reduced density ratio, $r$
    
    \item one-dimensional stellar evolution models suggest that the average reduced density ratio, $r$, should vary as a function of mass and metallicity, with stars of lower masses and lower metallicities having smaller ratios.
    
    \item one-dimensional stellar evolution models suggest that the average reduced density ratio, $r$, is not strongly dependent on the assumed parameterized thermohaline mixing model (e.g. Kippenhahn versus Brown) nor the efficiency parameter $\alpha$
    
    \item one-dimensional stellar evolution models suggest that the average reduced density ratio, $r$, occupies only a small range of the parameter space formally unstable to thermohaline mixing $0<r<1$, although most individual models do sample this whole range at some point.  
    %The real range covered in low-mass upper RGB stars is (a lot/ a little) compared to the range of simulations that have been done. These stellar evolution models suggest that R0 should depend strongly on composition and only weakly on stellar mass
    
    \item observations suggest that the amount of mixing is strongly correlated with the reduced density ratio $r$ as predicted in three dimensional simulations. 
    
    \item observations indicate that as the reduced density ratio $r$ increases, the mixing rate decreases, consistent with hydrodynamic prescriptions but not magnetohydrodynamic models
    
    \end{itemize}
    
     this suggests that (things about 3D models)
    
    magnetized thermohaline is/is not a good predictive theory for mixing in low-mass upper RGB stars

    
    If this is the case, should use this theory to predict cool stuff for lithium rich star production, mass transfer systems, etc. 
    
    Cool suggestions for other projects that explore all the things boat boy has been worried about

