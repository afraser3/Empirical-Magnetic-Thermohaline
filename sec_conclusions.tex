Thermohaline mixing has long been considered a good candidate to explain the evolution of the surface chemistry of low-mass upper red giant branch stars. We show here that:
\textbf{[I suggest enumerate rather than itemize here - MJ] JT: i'm worried numbers create a hierarchy that bullet points don't, but I'm not going to die on this hill}
\begin{itemize}
    \item three-dimensional simulations of the thermohaline instability suggest that mixing rates should strongly depend on the reduced density ratio, $r$
    
    \item one-dimensional stellar evolution models suggest that the average reduced density ratio, $r$, should vary as a function of mass and metallicity, with stars of lower masses and lower metallicities having smaller $r$.
    
    \item one-dimensional stellar evolution models suggest that the average reduced density ratio, $r$, is not strongly dependent on the assumed parameterized thermohaline mixing model (e.g. Brown or Kippenhahn regardless of $\alpha_{\rm th}$).
    
    \item one-dimensional stellar evolution models suggest that the average reduced density ratio, $r$, occupies only a small range of the parameter space formally unstable to thermohaline mixing $0<r<1$.
    %Note that each model samples larger $r$ at early times when the zone is developing and also the edges of the zone have $r \sim 1$, but we don't ever say this earlier in the paper.
    %The real range covered in low-mass upper RGB stars is (a lot/ a little) compared to the range of simulations that have been done. These stellar evolution models suggest that R0 should depend strongly on composition and only weakly on stellar mass
    
    \item observations suggest that the amount of mixing is strongly correlated with the reduced density ratio $r$ (read: \emph{there is a correlation}, it's not just a mess). This qualitatively matches predictions from three dimensional simulations. {\bf (This should probably be higher up; the lede is a bit buried here) JT right now they're in a logical flow summarizing the paper for people who skipped right to the end, but there's definitely an argument for important things first}
    
    \item observations indicate that the mixing rate and reduced density ratio $r$ are inversely correlated, which is consistent with hydrodynamic prescriptions but not magnetohydrodynamic prescriptions (read: not only are they correlated, they're correlated in the way that hydro models predict!).
    
    \end{itemize}
    %


Most generally, we excited to see that this framework of connecting the observations of extra mixing to the fluid simulations of the thermohaline instability through the medium of one dimensional stellar evolution models seems to be a feasible one. It will in general allow more rigorous exploration into whether red giant branch extra mixing should be associated with the thermohaline instability than ever before. It will also allow information to feed back into the theoretical simulations, as observations become able to constrain (? the efficiency of the mixing, the relevant fluid parameters, and potentially even the magnetic fields of interest in this regime). 

However, we stress that this is mostly a proof of concept, and that there is significant work still be done on all parts of this method. On the data side, we have so far only looked at the change in [C/N] in a fairly restricted set of stars. Looking at higher or lower masses, using a variety of mixing diagnostics, probing stars including globular clusters, binary stars, and dwarf galaxies where the base composition and mixing history may be different could all be illuminating. There is also the potential to look at the timing of the extra mixing, as well as the rate at which it occurs over time, and whether that can be connected to the sorts of (asteroseismic) diagnostics that probe the interior structure of a star, including its densities \citep{KjeldsenBedding1995}, chemical discontinuities \citep{Verma2017}, rotation \citep{Gehan2018}, and magnetic fields \citep{Bugnet2021}. 

On the modeling side, we have attempted to explore a reasonable range of possible choices that could impact our conclusions, but we do not claim to have been exhaustive. It would be very interesting to test, for example, whether
    
    this suggests that (things about 3D models)
    
    magnetized thermohaline is/is not a good predictive theory for mixing in low-mass upper RGB stars

    
    If this is the case, should use this theory to predict cool stuff for lithium rich star production, mass transfer systems, etc. 
    
    Cool suggestions for other projects that explore all the things boat boy has been worried about (\partyparrot).

