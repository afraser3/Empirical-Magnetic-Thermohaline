\documentclass[linenumbers,twocolumn]{aastex62}
\usepackage{enumitem}
%\usepackage{courier}
\graphicspath{{./}{figures/}}
\usepackage[T1]{fontenc}
\usepackage{epsfig}
\usepackage{epstopdf}
\epstopdfsetup{update}
\usepackage{natbib}
\usepackage{amssymb}
\usepackage{amsbsy}
\usepackage{natbib}
\usepackage{subfigure}
\usepackage[mathcal]{euscript}
\usepackage{float}
\usepackage{amsmath}
\usepackage{tabularx}
\usepackage{xspace}
\usepackage{enumitem}
\usepackage{xcolor}
\usepackage{comment}
%\usepackage{appendix}
\usepackage{ulem}

\newcommand{\partyparrot}{ \textcolor{green}{pa}\textcolor{red}{rty} \textcolor{blue}{pa}\textcolor{orange}{rr}\textcolor{purple}{ot} }


\newcommand{\blu}{\textcolor{blue} }
\newcommand{\red}{\textcolor{red} }
\newcommand{\grn}{\textcolor{green} }
\newcommand{\cya}{\textcolor{cyan} }
\newcommand{\pink}{\textcolor{magenta} }


\usepackage{ulem,xspace}
\newcommand{\paul}[1]{\textbf{\textcolor{blue}{#1}}}

\newcommand{\crossed}[1]{\red{\xout{#1}}} %querdurchgestrichen
\newcommand{\commenta}[1]{{\red{[#1]}}}
\newcommand{\change}[2]{\crossed{#1} \paul{#2}} %crosses out old text and adds text suggested as replacement
\newcommand{\changeC}[3]{\crossed{#1} \paul{#2} \textit{\red{[#3]}}}
\newcommand{\commentR}[2]{\textbf{\textcolor{blue}{#1}} \textit{\red{[#2]}}}
\newcommand{\commentU}[2]{\pink{\uwave{#1}} \textit{\red{[#2]}}}


%\newcommand{\paul}[1]{{\blu{#1}}}
\newcommand{\DP}{$\Delta P$\xspace}
\newcommand{\DPi}{$\Delta \Pi_1$\xspace}
\newcommand{\kms}{km~s$^{-1}$}
\newcommand{\vsini}{\ensuremath{v \sin{i}}}
%\newcommand{\msun}{M$_\sun$}
\newcommand{\tess}{{\it TESS}}
\newcommand{\corot}{{\it CoRoT}}
\newcommand{\jwst}{{\it JWST}}
\newcommand{\kepler}{{\it Kepler}}
\newcommand{\ktwo}{{\it K2}}
\newcommand{\hst}{{\it HST}}
\newcommand{\msun}{$M_{\odot}$}
\newcommand{\rsun}{$R_{\odot}$}
\newcommand{\lsun}{$L_{\odot}$}
\newcommand{\re}{$R_{\oplus}$}
\newcommand{\me}{$M_{\oplus}$}
\newcommand{\rj}{$R_{\textrm{\scriptsize Jup}}$}
\newcommand{\mj}{$M_{\textrm{\scriptsize Jup}}$}
\newcommand{\ms}{m~s$^{-1}$}
\newcommand{\gaia}{\textit{Gaia}}
\newcommand{\spherex}{\textit{SPHEREx}}

\newcommand{\teff}{\mbox{$T_{\rm eff}$}}
\newcommand{\logg}{\mbox{$\log g$}}
\newcommand{\feh}{\mbox{$\rm{[Fe/H]}$}}
\newcommand{\fbol}{\mbox{$f_{\rm bol}$}}
\newcommand{\ctwelvecthirteen}{$^{12}$C/$^{13}$C\xspace}
\newcommand{\hethree}{$^3$He\xspace}

\newcommand{\grad}{\ensuremath{\nabla}}
\newcommand{\gradrad}{\ensuremath{\nabla_{\rm{rad}}}}
\newcommand{\gradad}{\ensuremath{\nabla_{\rm{ad}}}}
\newcommand{\gradmu}{\ensuremath{\nabla_{\mu}}}
\newcommand{\gradL}{\ensuremath{\nabla_{\mathrm{L}}}}
\newcommand{\gradT}{\ensuremath{\nabla_{\mathrm{T}}}}
\newcommand{\brunt}{{Brunt--V\"{a}is\"{a}l\"{a}}}
\newcommand{\Pran}{\ensuremath{\mathrm{Pr}}}
\newcommand{\Dth}{\ensuremath{D_\mathrm{th}}}
\newcommand{\Numu}{\ensuremath{\mathrm{Nu}_\mu}}

%\newcommand{\kiauhoku}{K$\bar{\rm i}$auh$\bar{\rm o}$k$\bar{\rm u}$}
\newcommand{\kiauhoku}{\texttt{kiauhoku}} % I've adopted DFM's practice of using \texttt for software names - zach

\citestyle{aa}
\bibpunct{(}{)}{;}{a}{}{,}

\begin{document}

\title{Observed Extra Mixing Trends in Red Giants are Reproduced by the Reduced Density Ratio in Thermohaline Zones}

\author[0000-0003-4323-2082]{Adrian E. Fraser}
\altaffiliation{These authors contributed equally to this work}
\affiliation{Department of Applied Mathematics, Baskin School of Engineering, University of California, Santa Cruz, CA 95064, USA}
\affiliation{Kavli Institute for Theoretical Physics, University of California, Santa Barbara, CA 93106, USA}

\author[0000-0002-8717-127X]{Meridith Joyce}
\altaffiliation{These authors contributed equally to this work}
\affiliation{Lasker Fellow}
\affiliation{Space Telescope Science Institute,
3700 San Martin Drive,
Baltimore, MD 21218, USA}
\affiliation{Kavli Institute for Theoretical Physics, University of California, Santa Barbara, CA 93106, USA}
%\affiliation{\textdagger}

\author[0000-0002-3433-4733]{Evan H. Anders}
\altaffiliation{These authors contributed equally to this work}
\affiliation{CIERA, Northwestern University, Evanston IL 60201, USA}
\affiliation{Kavli Institute for Theoretical Physics, University of California, Santa Barbara, CA 93106, USA}

\author[0000-0002-4818-7885]{Jamie Tayar}
\altaffiliation{These authors contributed equally to this work}
\affiliation{NASA Hubble Fellow}
\affiliation{Department of Astronomy, University of Florida, Bryant Space Science Center, Stadium Road, Gainesville, FL 32611, USA }
\affiliation{Institute for Astronomy, University of Hawai‘i at Mānoa, 2680 Woodlawn Drive, Honolulu, HI 96822, USA}
\affiliation{Kavli Institute for Theoretical Physics, University of California, Santa Barbara, CA 93106, USA}

\author[0000-0001-5048-9973]{Matteo Cantiello}
\affiliation{Center for Computational Astrophysics, Flatiron Institute, New York, NY 10010, USA}
\affiliation{Department of Astrophysical Sciences, Princeton University, Princeton, NJ 08544, USA}
\affiliation{Kavli Institute for Theoretical Physics, University of California, Santa Barbara, CA 93106, USA}

\correspondingauthor{Adrian Fraser}
\email{adfraser@ucsc.edu}
\correspondingauthor{Meridith Joyce}
\email{mjoyce@stsci.edu}
\correspondingauthor{Evan Anders}
\email{evan.anders@northwestern.edu}
\correspondingauthor{Jamie Tayar}
\email{jtayar@ufl.edu}

\begin{abstract}
% %%%% MC: Attempt to clean up the abstract. Feel free to revert / take apart. Old version below
% Observations show the presence of an almost ubiquitous extra mixing in low-mass upper giant branch stars.  
% A possible origin for this extra mixing is a thermohaline instability induced by a mean molecular weight inversion in otherwise stably-stratified zones. 
% Stellar evolution models include prescriptions for thermohaline mixing that are informed by analytical calculations and three-dimensional fluid dynamics simulations, 
% but a direct comparison between these models and observations is challenging. This is because current computational resources do not allow modeling thermohaline mixing under the relevant stellar conditions, 
% and only extrapolations of lower-resolution calculations are available. Moreover stellar interiors are complex regions, with likely multiple concurrent fluid dynamics processes interacting. 
% Here we take advantage of carbon to nitrogen measurements from the SDSS-IV APOGEE survey of stars to put empirical constraints on the efficiency of mixing across a relatively large range of masses and metallicities. 
% We identify the reduced density ratio as an important parameter to compare the observed amount of extra mixing on the upper giant branch, and the predicted trends from three-dimensional fluid dynamics simulations.
% The reduced density ratio is a function of the ratio of thermal and compositional stratification, as well as of the ratio of the thermal and compositional diffusivities. 
% We show that the observed amount of extra mixing is strongly correlated with the reduced density ratio. Since this trend is also reproduced in three-dimensional models of thermohaline instability, our results 
% give support to a thermohaline mixing origin for the observed surface abundances change. 




% It has been widely assumed that the extra mixing observed in low-mass giant branch stars is related to the thermohaline instability that can be induced by mean molecular weight inversions in otherwise stably stratified zones. However, direct comparison between numerical simulations and observations has been challenging. %It was suggested that their extra mixing might be due to thermohaline convection, which relies on ( mean molecular weight inversions in otherwise stably stratified zones). However, previous attempts to match the observed pattern quantitatively using one dimensional stellar evolution models have usually required some sort of ad hoc increase in the efficacy of the thermohaline mixing, and recent three dimensional simulations  have suggested contradictory results on how the amount of mixing should depend on the fluid parameters of the star in question. 
% Here we compute the relevant fluid parameters from MESA models, including the reduced density ratios, for stars at a range of masses and metallicities. Combining these modeling results with carbon-to-nitrogen measurements from the SDSS-IV APOGEE survey, \textbf{we provide a framework to qualitatively compare \red{models derived from} three dimensional fluid simulations to the amount of extra mixing on the upper giant branch.}  We show that stars with available mixing data tend on average to be at relatively low density ratios, which informs the regime which future simulations should attempt to probe. We also show that the amount of extra mixing is strongly correlated with the reduced density ratio, which \red{is consistent with the expectations of thermohaline mixing.} %suggests that this framework is valid for exploring the physics of this mixing. 
% Finally, we show that there is increased mixing at low reduced density values, which is consistent with current hydrodynamical models of the thermohaline instability.  

Observations show an almost ubiquitous presence of extra mixing in low-mass upper giant branch stars. The most commonly invoked explanation for this is the thermohaline instability. One dimensional stellar evolution models include prescriptions for thermohaline mixing, but our ability to make direct comparisons between models and observations has thus far been limited.
Here, we propose a new framework to facilitate direct comparison:
Using carbon to nitrogen measurements from the SDSS-IV APOGEE survey as a probe of mixing and a fluid parameter known as the \textit{reduced density ratio} from one dimensional stellar evolution programs, we compare the observed amount of extra mixing on the upper giant branch to predicted trends from three-dimensional fluid dynamics simulations. 
By applying this method, we are able to place empirical constraints on the efficiency of mixing across a range of masses and metallicities. 
We find that the observed amount of extra mixing is strongly correlated with the reduced density ratio and that trends between reduced density ratio and fundamental stellar parameters are robust across choices for modeling prescription. We show that stars with available mixing data tend to have relatively low density ratios, which should inform the regimes selected for future simulation efforts. Finally, we show that there is increased mixing at low values of the reduced density ratio, which is consistent with current hydrodynamical models of the thermohaline instability. 
The introduction of this framework sets a new standard for theoretical modeling efforts, as validation for not only the amount of extra mixing, but trends between the degree of extra mixing and fundamental stellar parameters is now possible.
%
\end{abstract}

% s it worth including one last punchline sentence saying “this is a big freaking deal because …” like “We hope this framework will provide a new target for theoretical modeling efforts: not only can new mixing prescriptions be compared against observations in terms of the overall amount of extra mixing, but also trends”


\keywords{stellar evolution, stellar abundances, abundance ratios, stellar interiors, red giant branch, red giant bump, Dredge-up}

\section{Introduction  } 
\label{sec:intro}
\setcounter{footnote}{0}
%
%
%
Observations of globular clusters and low-metallicity field stars show significant changes in the abundance ratios of elements known to be sensitive to mixing, including \ctwelvecthirteen, lithium, and [C/N], as a star evolves up the red giant branch \citep{Carbon1982, Pilachowski1986, Kraft1994, Shetrone2019}. These changes occur around the red giant branch bump (RGBB) and are largest in the most metal-poor stars \citep[e.g.][]{Gratton2000}. Large samples of stars that can be used to trace this mixing are now available from a variety of spectroscopic surveys, including GALAH \citep{buder2019}, APOGEE \citep{DR17}, GAIA-ESO \citep{Magrini2021b}, and others. However, observed surface abundance trends are in tension with standard theoretical stellar evolution models, which predict that the surface chemistry should not evolve in this regime.

As low-mass stars ascend the red giant branch, they undergo a series of mixing and homogenizing events as their interior burning and energy transport zones interact. Near the base of the red giant branch, the surface convection zone reaches its deepest level of penetration into the stellar interior, leaving behind a chemical discontinuity from which it recedes in subsequent evolution. This inflection in the convection zone's movement is known as the ``first dredge-up.'' The red giant branch bump occurs when the outward-propagating hydrogen burning shell encounters this chemical discontinuity, triggering a structural realignment in which the star's core contracts and the luminosity drops, causing a disruption to the otherwise monotonic increase in luminosity along the red giant branch \citep{Christensen-Dalsgaard:2015}. In one dimensional (1D) stellar models, the sensitivity of the RGBB to physical assumptions makes it a powerful diagnostic of interior mixing processes \citep[e.g.][]{Joyce2015, Khan2018}. 
%
However, in standard models of red giant stars, there is no mixing between the hydrogen-burning shell and the overlying convective envelope after the first dredge-up, and no change in surface abundances is predicted in this regime. This is in direct conflict with abundance trends found in observations.

The most widely studied candidate mechanism for rectifying this discrepancy is thermohaline mixing, identified in this context by \citet{charbonnel_thermohaline_2007} and others. As the hydrogen-burning shell moves into the region chemically homogenized by the  first dredge-up, the $^3$He($^3$He, 2p)$^4$He reaction creates an inversion of the mean molecular weight $\mu$. While this $\mu$ inversion is insufficient to generate a convective region (c.f. \citealt{CantielloLanger2010}), these conditions give rise to the 
\textit{thermohaline instability}, a phenomenon perhaps best known in the context of salt water in Earth's oceans \citep{Stern1960,baines_gill_1969}. 

Thermohaline mixing is a double-diffusive phenomenon present in fluids that have different diffusivities for heat and chemical composition which in turn make opposing contributions to the vertical density gradient \citep{Turner:1974}. Thermohaline mixing occurs in Ledoux-stable regions that have stably stratified temperature gradients but unstable mean molecular weight stratification \citep[see][for a full review]{garaud_DDC_review_2018}. This process may facilitate the vertical mixing of elements between the hydrogen-burning shell and the stellar convective envelope, thus producing measurable changes in the surface mixing diagnostics after the first dredge-up. 

Given that the physical conditions required to trigger the thermohaline instability are in place at around the same time that extra mixing has been observed in red giant stars \citep[e.g.][]{Lagarde2015}, most authors have assumed that all of the observed extra mixing can be attributed to the thermohaline instability \citep[e.g.][]{Kirby2016, Charbonnel2020, Magrini2021a}. However, this connection has also been questioned for a number of reasons. 

First, reproducing the observed amounts of mixing in this regime with 1D models requires the adoption of much higher efficiency parameters than most fluid simulations would suggest are reasonable or physical \citep{Denissenkov2010thermohaline, denissenkov_merryfield_2011, traxler_etal_2011, brown_etal_2013}. Questions have likewise been raised about whether the evolutionary timing of the observed extra mixing is truly consistent with thermohaline models \citep[see e.g.][]{Angelou2015, Henkel2017, TayarJoyce22}.

There has also been some debate about how the fluid instability should be parameterized in one dimension, and authors have proposed a variety of different prescriptions informed by numerical simulations \citep[e.g.~][]{traxler_etal_2011,brown_etal_2013}. However, RGB stars have much lower ratios of kinematic viscosity to thermal diffusivity than simulations can reach, of the order $\mathrm{Pr} \sim 10^{-6}$, whereas modern fluid simulations can only probe as low as $10^{-2} - 10^{-3}$. This has generated skepticism about whether trends from simulations can be accurately extrapolated into stellar regimes. Likewise, models of thermohaline instability that include the presence of a relatively low-amplitude magnetic field can result in much larger diffusivities \citep{harrington}, raising the question of whether earlier prescriptions that neglect magnetic fields may be missing key physics. 



Given both these observational and theoretical questions, the development of a framework through which we can determine whether signatures from true stellar conditions (observations, $\mathrm{Pr} = 10^{-6}$) are qualitatively consistent with fluid models is \textbf{timely and imperative.} 
%\blu{pretty sure timely and imperative is also in the conclusion right now lol. also strikeout Pr if you decide not to define it above.}
In this paper, we put forth such a framework: one that allows not only the calibration of individual mixing parameterizations, but also comparison between mixing models. We demonstrate \a robust and model-agnostic means of relating the non-dimensional fluid parameters relevant to thermohaline mixing to the observed mixing around the RGB bump and show that this correlation is indeed qualitatively consistent with 1D prescriptions of thermohaline mixing informed by 3D simulations. Further, while previous work (e.g.~Charbonnel \& Zahn) have used the measurements of the \textit{overall amount} of extra mixing to tune the \textit{overall efficiency} of thermohaline mixing prescriptions, our framework allows us to use trends in extra mixing as a function of fundamental stellar parameters to probe trends predicted by various prescriptions.

This paper is organized as follows: we begin by summarizing the formalism and stellar structure quantities relevant to thermohaline mixing (Sec.~\ref{sec:formalism}). This is followed by a description of various 1D mixing prescriptions commonly adopted in stellar evolution calculations (Sec.~\ref{sec:parameterizations}). We then introduce a suite of 1D MESA simulations and calculate the relevant fluid parameters in the thermohaline region for a range of mass and metallicity assumptions (Secs.~\ref{sec:mesa_experiment} and \ref{sec:mesa_results}). Finally, we compare an observational proxy of extra mixing, the decrease in [C/N] near the RGBB, to theoretical trends predicted by existing 1D thermohaline mixing prescriptions (Sec.~\ref{sec:obs} and Sec.~\ref{sec:punchline}). Our results and their implications are discussed in Sections \ref{sec:punchline} and \ref{sec:conclusions}. 

\section{Thermohaline Formalism }
\label{sec:formalism}
The instability driving thermohaline mixing requires a Ledoux-stable inversion of the mean molecular weight $\mu$ in the presence of a stable temperature gradient. 
Expressed in terms of typical stellar structure variables, the stability of the temperature gradient is given by the Schwarzschild criterion:
\begin{equation} \label{eq:Schwarzschild}
    \gradrad - \gradad < 0,
\end{equation}
\textcolor{red}{[Evan: paraphrase this sentence]} where the temperature gradient $\grad \equiv d \ln P / d \ln T$ (pressure $P$ and temperature $T$) is $\gradad$ for an adiabatic stratification and $\gradrad$ if all the flux is carried radiatively. 
The statement that the temperature gradient is strong enough to ensure stability by the Ledoux criterion (i.e.~the despite the inversion of the mean molecular weight is
\begin{equation} \label{eq:Ledoux}
    \gradrad - \gradad - \frac{\phi}{\delta}\gradmu < 0.
\end{equation}
\textcolor{red}{[paraphrase this sentence]} The Ledoux criterion includes the effects of the composition gradient $\gradmu = d\ln\mu/d\ln P$ (mean molecular weight $\mu$), where $\delta = -(\partial \ln \rho / \partial \ln T)_{P,\mu}$ and $\phi = (\partial \ln \rho / \partial \ln\mu)_{P,T}$ (density $\rho$).

The stabilizing influence of the temperature gradient relative to the destabilizing influence of the $\mu$ gradient is often given in terms of the density ratio $R_0$, defined as
\begin{equation} \label{eq:R0}
    R_0 \equiv \frac{\grad - \gradad}{\frac{\phi}{\delta} \gradmu},
\end{equation}
where $R_0 < 1$ implies the $\mu$ gradient is sufficiently strong to drive convection, and $R_0 > 1$ implies the fluid is stably-stratified (i.e.~no convection) \textcolor{red}{[I'm being REAL careful not to say $R_0 > 1$ is Ledoux-stable, because of the $\grad$ in Eq.~\eqref{eq:R0} vs the $\gradrad$ in Eq.~\eqref{eq:Ledoux}]}. 
As reviewed by [cite Garaud review 2018], fluids with $R_0 > 1$ can be prone to double-diffusive instabilities whenever the thermal diffusivity $\kappa_T$ is greater than the compositional diffusivity $\kappa_\mu$. Specifically, the instability driving thermohaline mixing acts whenever
\begin{equation} \label{eq:R0_condition}
1 < R_0 < 1/\tau,
\end{equation}
[cite Stern 1960, I'm pretty sure] where
\begin{equation} \label{eq:tau}
    \tau \equiv \kappa_\mu/\kappa_T.
\end{equation}
Note that typical values of $\tau$ in stellar radiation zones are $10^{-6}$ or smaller, meaning even very slightly destabilizing $\mu$ gradients [how do I make it more obvious that this means ``big $\mu$ over small $\mu$"?] can drive thermohaline mixing, even when the temperature gradient is far from being convectively unstable.

\section{Parameterized Thermohaline Models }
\label{sec:parameterizations}
%
Equation \eqref{eq:r_condition} 
can be readily evaluated at any radial location in a model star generated with a 1D stellar structure and evolution program. However, predicting the efficiency of thermohaline mixing is much more challenging. A diffusive approximation is commonly taken for the turbulent mixing of chemicals such that the total mixing of chemical species is given by the sum of the molecular diffusivity and a turbulent mixing coefficient, $\Dth$. \textbf{This diffusion coefficient, $\Dth$, relates the rate of chemical mixing to the chemical composition gradient, and is broadly what is meant when referring to ``mixing efficiency" throughout this paper.} This quantity can be converted to the compositional Nusselt number discussed in the fluid dynamics literature, $\Numu$, using the formula

\begin{equation} \label{eq:Dth_from_Nu}
    \Dth = (\Numu - 1)\kappa_\mu.
\end{equation}

We call any model that predicts $\Dth$ as a function of stellar structure variables (e.g.~gradients and molecular diffusivities of chemicals and heat) a parameterized mixing model or mixing prescription. 
Efforts to develop thermohaline mixing prescriptions for use in models of stellar interiors date back many decades, see \citet{garaud_DDC_review_2018} for a full review. 
Such mixing prescriptions have been implemented in a variety of 1D stellar evolution programs \citep[see][and references therein]{lattanzio_etal_2015}, enabling studies of the effects of thermohaline mixing in stars across the Hertzprung-Russell diagram. 
Here, we briefly summarize the most commonly used and more recently developed prescriptions.

The \textit{de facto} thermohaline mixing model used in MESA (first described in \citealt{CantielloLanger2010} and implemented for public use in \citealt{mesa2}) is commonly referred to as the ``Kippenhahn model'' and was originally derived by \citet{Ulrich1972} and \citet{kippenhahn_etal_1980}.
Using arguments based on dimensional grounds and assumptions about the shapes and motions of discrete fluid parcels, they derived a mixing efficiency of the form

\begin{equation} \label{eq:Dth-kipp}
    \Dth = C_t \kappa_T R_0^{-1},
\end{equation}
\citep[see Eq.~(5) of][]{charbonnel_thermohaline_2007}
where $C_t$ is a free parameter, with plausible values ranging from $C_t = 658$ \citep{Ulrich1972} to $C_t = 12$ \citep{kippenhahn_etal_1980}. 
We note that Eq.~\eqref{eq:Dth-kipp} predicts finite mixing for $r \geq 1$ ($R_0 \geq 1/\tau$), even though thermohaline mixing is formally stabilized for these parameters.

Nevertheless, Eq.~\eqref{eq:Dth-kipp} is implemented in MESA as
\begin{equation} \label{eq:Dth-kipp-MESA}
    \Dth = \frac{3}{2} \alpha_{\rm{th}} \frac{K}{\rho C_P}R_0^{-1}
\end{equation}
\citep[see Eq.~(14) of][]{mesa2}. 
Here, $\alpha_{\rm{th}}$ is a dimensionless efficiency parameter related to $C_t$ by $C_t = 3\alpha_{\rm{th}}/2$, $K$ is the radiative conductivity, $\rho$ is the density, and $C_P$ is the specific heat at constant pressure, with $\kappa_T = K/(\rho C_P)$. 
The green curve in Fig.~\ref{fig:parameterization_compare} shows $\Dth/\kappa_\mu$ vs.~$r$ calculated according to Eq.~\eqref{eq:Dth-kipp-MESA} for $\tau = 10^{-6}$, which is a representative value for the thermohaline-unstable region of RGB stars, and the same $\alpha_{\rm{th}} = 2$ assumed in \citet{CantielloLanger2010}.

In addition to tension regarding the choice of \textbf{model parameters (e.g.~$\alpha_\mathrm{th}$) controlling overall mixing efficiency within a given 1D prescription} \citep[see e.g.][for further discussion]{Ulrich1972, kippenhahn_etal_1980, charbonnel_thermohaline_2007, CantielloLanger2010,traxler_etal_2011}, there have also been questions about the appropriate trends in \textbf{mixing efficiency} as a function of fluid parameters \textbf{(i.e.~how $\Dth$ should depend on quantities like $r$ and $\mathrm{Pr}$)} and therefore the stellar structure variables on which they depend \citep{garaud_DDC_review_2018}.
Thus, recent work has sought to refine these mixing prescriptions by performing numerical experiments with multi-dimensional simulations to more accurately parameterize mixing efficiency \citep{Denissenkov2010thermohaline,traxler_etal_2011}. 
\citet{traxler_etal_2011} and \citet{brown_etal_2013} performed 3D hydrodynamic simulations across a broad range of parameters. 
Not only did they find orders of magnitude less mixing than what is predicted by the Kippenhahn model with the \textbf{model} parameter required in \citet{charbonnel_thermohaline_2007} to find agreement with observations ($C_t = 1000$), they also developed new mixing prescriptions that fit their simulations much more closely. 
In the case of \citet{traxler_etal_2011}, the authors derived a mixing law by fitting an
analytic function 
of the form

\begin{equation} \label{eqn:trax_model}
   \Dth = \kappa_{\mu}\sqrt{\frac{\mathrm{Pr}}{\tau}}\left(a e^{-br}[1 - r]^c\right),
\end{equation}
to their simulation results,
where 

\begin{equation} \label{eq:Prandtl}
    \mathrm{Pr} = \frac{\nu}{\kappa_T}
\end{equation}
is the Prandtl number, with $\nu$ the kinematic viscosity,
%
and $a$, $b$, and $c$ are constants which they fit to data. 

While \citet{traxler_etal_2011} clearly showed their simulations are inconsistent with the mixing efficiency $\Dth$ implied by the Kippenhahn model with $\alpha_{\rm{th}}, C_t \sim 10^2-10^3$, it is important to note that their simulations generally explored $\mathrm{Pr}, \tau \sim 10^{-1}$, whereas these fluid parameters are generally of the order $10^{-6}$ in the radiative interiors of RGB stars. 
Thus, a fair question is whether mixing efficiency might increase to these larger values as $\mathrm{Pr}$ and $\tau$ approach $10^{-6}$. 
However, \citet{traxler_etal_2011} varied these parameters by an order of magnitude in their simulations, and investigated trends in $\Dth$. They found that mixing should not increase in this fashion, as indicated by the dependence of $\Dth$ on $\mathrm{Pr}$ and $\tau$ in Eq.~\eqref{eqn:trax_model}, which makes an argument that these models can be made to fit the observational data difficult to justify. 

\citet{brown_etal_2013} note that the model in Eq.~\eqref{eqn:trax_model} performs well at high $R_0$, but underestimates mixing at low $R_0$, particularly in the stellar regime of low Pr and $\tau$.
They develop a semi-analytical model for thermohaline mixing,

\begin{equation}
    \Dth = \kappa_{\mu}C^2\frac{\lambda^2}{\tau l^2(\lambda + \tau l^2)},
    \label{eqn:brown_model}
\end{equation}
where $\lambda$ is the growth rate of the fastest-growing linearly unstable mode, $l$ is its horizontal wavenumber, and $C \approx 7$ was fit to data from 3D hydrodynamic simulations.
Both $\lambda$ and $l$ are functions of $\mathrm{Pr}$, $\tau$, and $R_0$, and are obtained by finding the roots of a cubic and quadratic polynomial (their Eqs.~19 and 20).
The orange curve in Fig.~\ref{fig:parameterization_compare} shows $\Dth/\kappa_\mu$ vs.~$r$ calculated according to Eq.~\eqref{eqn:brown_model} for $\mathrm{Pr} = \tau = 10^{-6}$, representative values for the thermohaline-unstable regions of RGB stars. 
Note that $\Dth/\kappa_\mu \to 0$ as $r \to 1$ as expected, since the thermohaline instability becomes stable for $r \geq 1$.
We see that Eq.~\eqref{eq:Dth-kipp-MESA} with $\alpha_{\rm{th}} = 2$ agrees with this prescription for some values of $r$, suggesting that significantly larger values of $\alpha_{\rm{th}}$ are not consistent with 3D hydrodynamic simulations. 
While the general dependence of $\Dth/\kappa_\mu$ on $r$ is significantly different between these two models, they do both feature monotonically decreasing values of $\Dth/\kappa_\mu$ versus $r$. 
This prescription is implemented in MESA and has since been used in \citet{bauer_bildsten_2019} and other works. 

\citet{harrington} extended the work of \citet{brown_etal_2013} by performing 3D magnetohydrodynamic (MHD) simulations of thermohaline mixing with initially uniform, vertical magnetic fields of various strengths to approximate the effects of magnetic fields from external processes including, for instance, a global dipole field or a large-scale magnetic field left behind by a dynamo acting in the receding convective envelope. 
They found that magnetism strictly increases mixing efficiency, sometimes dramatically.
They developed a mixing prescription that accounts for this effect by building on the model of \citet{brown_etal_2013}.
The strength of the magnetic field is introduced into their model through their parameter $H_B$, which is proportional to the square of the magnetic field strength and depends on other stellar structure variables \citep[see Eq.~19 of][]{harrington}.
Their mixing prescription is of the form

\begin{equation} \label{eq:harrington_model}
    \Dth = \kappa_{\mu}K_B\frac{w_f^2}{\tau (\lambda + \tau l^2)},
\end{equation}
where $w_f$ is obtained by solving a quartic polynomial that includes the magnetic field strength through $H_B$, and $K_B \simeq 1.24$ is directly related to the constant $C$ in Eq.~\eqref{eqn:brown_model}.

This mixing prescription agreed remarkably well with their 3D simulations, which were limited to $r = 0.05$ but ranged in magnetic field strength over several orders of magnitude.
The prescription, which has not yet been implemented in MESA at the time of this writing, has two asymptotic limits, one where $\hat{w}_f^2 \propto B_0^2$ when the magnetic field strength $B_0$ is large, and one which reduces to the model of \citet{brown_etal_2013} when $B_0$ is small.

The purple curve in Fig.~\ref{fig:parameterization_compare} shows $\Dth/\kappa_\mu$ vs.~$r$ calculated according to Eq.~\eqref{eq:harrington_model} for the same parameter choices as the orange curve, and with $H_B = 10^{-6}$, appropriate for the thermohaline zone of a 1.1 $M_\odot$ star at [Fe/H] = -0.2 and a magnetic field whose strength is $\mathcal{O}(100 \,\mathrm{G})$. 
Note that this magnetic field strength dramatically increases mixing efficiency relative to the hydrodynamic values, particularly at larger values of $r$, whereas the model predicts the same mixing as the Brown model for $r \lesssim 10^{-5}$. 
For larger values of $r$, the dependence of $\Dth/\kappa_\mu$ on $r$ is profoundly different than either of the hydrodynamic models, with $\Dth/\kappa_\mu$ increasing with $r$, even as the thermohaline instability approaches marginal stability as $r \to 1$.

\textbf{The dramatic enhancement in mixing efficiency predicted by this model for magnetic field strengths of even $\mathcal{O}(100\,\mathrm{G})$ presents a promising resolution to the tension discussed above, namely that 1D stellar evolution models can only reproduce observations by assuming that mixing is far more efficient than what is seen in 3D hydrodynamic simulations. 
%the mixing efficiency needed in 1D stellar evolution models in order to agree with observations is much larger than what is consistent with 3D hydrodynamic simulations. 
However, while their prescription may predict mixing efficiencies that are comparable in overall magnitude to that of the Kippenhahn model with $C_t \sim 10^3$, the two prescriptions yield qualitatively different trends in $\Dth$ vs $r$.} 
Given the variance of the predictions of these models, we focus in this paper on showing how observations can be used to suggest the \textit{trends} in mixing that models should hope to explain rather than on trying to calibrate \textbf{model parameters controlling} the overall mixing \textit{efficiency} for a particular model, as has been done before, in order to provide a framework in which we can distinguish between mixing prescriptions.

%are not.

\begin{figure}
    \centering
    \includegraphics[width=\columnwidth]{Nu_models_comparison.pdf}
    \caption{ 
    Prescriptions of the compositional diffusivity due to thermohaline mixing $D_{\rm th}$ normalized by the molecular diffusivity $\kappa_{\mu}$ are plotted against the reduced density ratio $r$. For each prescription, we use $\mathrm{Pr} = \tau = 10^{-6}$, consistent with the conditions in these regions of RGB stars.
    We plot two hydrodynamic models, the \citet{brown_etal_2013} model (orange) and the \citet{kippenhahn_etal_1980} model with $\alpha_{\rm th} = 2$ (green). In both cases, the mixing efficiency decreases with $r$.
    The \citet{harrington} model (HG19) is also shown; it includes magnetic fields, which cause mixing efficiency to increase with $r$ for these parameters.
    The plotted curve for the HG19 model depends on $H_B$, which depends on the stellar structure and magnetic field strength; the plotted value is characteristic of the structure in the thermohaline zone of a 1.1 $M_\odot$ star at [Fe/H] = -0.2 with a magnetic field whose strength is $\mathcal{O}(100 \,\mathrm{G})$.
    The purple-to-yellow color gradient plotted in the background denotes the range of $r$ values that we measure in our grid of 1D stellar evolution models, which are displayed in Fig.~\ref{fig:mesa_r_spread}.
    }
    \label{fig:parameterization_compare}
\end{figure}


\section{Stellar Evolution Models} %MC changed this from 'Mesa Configuration'
%previously called ``Mixing predicted by 1D models"
\label{sec:mesa_experiment}
%
%
%
We use MESA stable release version 21.12.21 to conduct 1D numerical simulations of thermohaline mixing for metallicities ranging from [Fe/H] $= -1.4$ to $0.4$ ($Z = 0.00068$ to $0.038$) and masses from 0.9 to 1.7 $M_{\odot}$ \textbf{(at resolutions of XX, YY?)}. We adopt the solar abundance scale of \citet{GrevesseSauval1998} and the corresponding opacities of \citet{IglesiasRogers1996}. We use an Eddington T-$\tau$ relation for the atmospheric surface boundary conditions.
We adopt the mixing length theory (MLT) prescription of \citet{Cox1980} with a fixed value of $\alpha_{\text{MLT}}= 1.6$ times the pressure scale height ($H_p$). We use the Ledoux criterion for convective stability and neglect the effects of convective overshoot \citep{Ledoux}. We use the \verb|pp_extras.net| nuclear reaction network, which contains 12 isotopes. Full details of our physical and numerical parameter choices are available on Zenodo\footnote{MESA inlists will be made available upon publication}. 

Simulations are evolved at $1.25\times$ the default mesh (i.e. structural) resolution and $2\times$ the default time resolution on the pre-main sequence and main sequence. Once the models ascend the red giant branch and reach a surface gravity $\log g \le 3$, resolutions are increased to $2\times$ the default spatial resolution and $10\times$ the default temporal resolution, respectively. Optimal resolution values were determined according to the convergence tests detailed in Appendix A.1. 

We study four grids of stellar evolution simulations with different thermohaline mixing prescriptions. One grid employs the \citet{brown_etal_2013} prescription with the recommended coefficient of 1, while the other three employ the \citet{kippenhahn_etal_1980} prescription with coefficients $\alpha_{\rm th} \in [0.1, 2, 700]$. \textbf{[motivations for these choices?]}
%dimensionless 

\subsection{Method for Extracting $r$ \textbf{(RTR for Meridith)}}
To measure $r$ in our MESA simulations, we first restrict to the appropriate evolutionary phase.
We rule out all models in which MESA does not detect thermohaline mixing within $m_{\rm max} \leq m_i < 1.1 m_{\rm max}$, where $m_i$ is the mass coordinate of the $i$th mass shell and $m_{\rm max}$ is the mass coordinate coinciding with the instantaneous peak of the nuclear energy generation measured in [erg/g/sec].
The thermohaline zone extends from a maximum mass coordinate $m_{\rm heavy}$ to a minimum mass coordinate $m_{\rm light}$ with stratification $\Delta m = m_{\rm heavy} - m_{\rm light}$.
We exclude the first 21 models in which the thermohaline zone spans at least 10 mass shells.
We then compute the evolution of $\Delta_{m}$ of the $j$th model, $\delta_m^j = \Delta_m^{j} - \Delta_m^{j-1}$ for the current model and the previous 20 models and compute $\langle \delta_m \rangle = (1/20)\sum_{j=-20}^0 \delta_m^j$; we expect $\langle \delta_m \rangle$ to be relatively large while the thermohaline zone is developing and small when it is in a relatively steady state.
We measure $\epsilon = |\langle \delta_m \rangle / \mathrm{max}(\Delta_m^j)|$; if $\epsilon < 5 \times 10^{-3}$, we consider the model to have reached a steady state of thermohaline mixing (classified as ``good'' or ``stable'') and we compute $r$.

To compute $r = (R_0 - 1)/(\tau^{-1} - 1)$, we take the volume average $\bar{r} = \sum r_i dV_i / \sum dV_i$ over a subset of mass bins $i$ of the thermohaline zone.
We volume-average $r$ over the mass range bounded by $m_{\rm{heavy}} + 0.1\Delta m  < m_i \leq m_{\rm heavy} + 0.43\Delta m$.
In the volume average, we set the volume element $dV_i = 4\pi r_i^2 \Delta r_i$ and perform integration using the composite trapezoidal rule as implemented in \texttt{NumPy}.
We stop measuring $r$ after we have collected measurements over 1000 models, which captures the behavior of the saturated thermohaline zone and its eventual merging with the convective shell.
For each stellar evolution simulation, we report the median of the volume-averaged $r$ over all of the stable models in which measurements were taken. 
Results are discussed in terms of the logarithm of this quantity, $\log_{10} r$.

\section{Results from Numerical Experiments }
\label{sec:mesa_results}
%
%
Figure \ref{fig:mesa_r_spread} compares results from four physical configurations describing thermohaline mixing in MESA: the upper left panel shows results from the Brown model; the remaining three show results from the Kippenhahn prescription with $\alpha_{\text{th}}$ varying as indicated. The reduced density ratio $\log_{10} r$ is shown as a function of mass and metallicity and indicated on the color bar and grid labels.
\sout{Simulations for which our algorithm could not measure a stable thermohaline zone are blocked out in grey and have no label; this happens at high mass and low metallicity.} \red{I think all models will have thermohaline with the corrected algorithms; TBD - EA}.
%

In all cases, the most notable trend is that $\log_{10} r$ decreases along the diagonal from high masses and metallicities (upper left) to low masses and metallicities (lower right). There is particularly high qualitative similarity between the Brown model and Kippenhahn model with $\alpha_{\text{th}} = 2$, which correspond to similar thermohaline mixing timescales. The case with the lowest mixing parameterization is the Kippenhahn $\alpha_{\text{th}} = 0.1$ case, and there the span of $\log_{10} r$ values is smallest. We also note that, unlike in the other three cases, $\log_{10} r$ does not scale precisely monotonically with either mass or [Fe/H] in the Kippenhahn $\alpha_{\text{th}} = 700$ case. While there is no clear relationship between the spread of $\log_{10} r$ values observed when using the Kippenhahn prescriptions and the values of $\alpha_{\text{th}}$ adopted in each, there is a clear relationship between the median values of $\log_{10} r$ and $\alpha_{\text{th}}$: the reduced density ratios are larger  when mixing is highly efficient (i.e. $t_{\mathrm th} << t_{\text{evol}}$). 
Most importantly, the overall behavior of $\log_{10} r$ as a function of mass and [Fe/H] is consistent regardless of the theoretical assumption adopted.
%
%The overall trends of $\log_{10} r$ vs.~[Fe/H] and mass are  
%regardless of the
This robustness across 1D thermohaline mixing model assumptions suggests that $r$ may be useful as a mixing diagnostic in physical data sets. We explore its application to observations subsequently.


\begin{figure*}
    \centering
    \includegraphics[width=\textwidth]{mesa_r_spread.pdf}
    \caption{The reduced density ratio $\log_{10} r$ is extracted as discussed in Section \ref{sec:mesa_experiment} for four grids of stellar models with differing prescriptions for thermohaline mixing. 
    Results for $\log_{10} r$ are shown as a function of stellar mass and metallicity [Fe/H], with high values of $\log_{10} r$ in brighter colors (yellow) and low values of $\log_{10} r$ in darker colors (purple). 
    The model name and mixing efficiency, $\alpha_{\text{th}}$ (where applicable) constitute the physical configuration and are indicated in the panel labels.}
    \label{fig:mesa_r_spread}
\end{figure*}

\section{Observed Mixing Signatures }
\label{sec:obs}
%On the upper giant branch, near the location of the red giant branch bump, there is an observed episode of mixing whose efficiency is observed to depend on the initial mass and metallicity of the star \citep[e.g.][]{Gratton2000}. This has been assumed to related to thermohaline mixing (see citecite) but this has been disputed for both theoretical (citecite) and observational reasons \citep{TayarJoyce2022} (citecite). 
%\begin{minipage}{1.0\textwidth}
\begin{table*}[tb]
\begin{center}
\caption{Observed extra mixing drops in bins of mass and metallicity, corrected for the  0.1456 dex of unmixing observed that we assume is due to systematic errors. We also include  the reduced density ratios  calculated for each of these bins using the variety of models discussed in Section \ref{sec:mesa_results}.}% \red{times 1000 for easier readability.}}
\begin{tabular}{rrrrrrrr}
\hline
\multicolumn{1}{l}{M} & \multicolumn{1}{l}{[Fe/H]} & \multicolumn{1}{l} {$\Delta$[C/N]$_{\rm APK, cor}$} & {$\Delta$[C/N]$_{\rm Shet, cor}$}  & \multicolumn{1}{l}{$r_{\rm Brown, 1}$} & \multicolumn{1}{l}{$r_{\rm Kip, 0.1}$} & \multicolumn{1}{l}{$r_{\rm Kip, 2}$} & \multicolumn{1}{l}{$r_{\rm Kip, 700}$} \\ \hline \hline
0.9 & -1.4 & ... & \multicolumn{1}{r}{0.73} & 0.00013 & 0.00012 & 0.00015 & 0.00066 \\ 
0.9 & -1.2 & ... & \multicolumn{1}{r}{0.67} & 0.00016 & 0.00013 & 0.00017 & 0.00073 \\ 
0.9 & -1.0 & ... & \multicolumn{1}{r}{0.48} & 0.00018 & 0.00014 & 0.00017 & 0.00078 \\ 
0.9 & -0.8 & 0.52 & \multicolumn{1}{r}{0.36} & 0.00020 & 0.00016 & 0.00020 & 0.00099 \\ 
0.9 & -0.6 & 0.29 & \multicolumn{1}{r}{0.27} & 0.00026 & 0.00018 & 0.00026 & 0.00140 \\ 
0.9 & -0.4 & 0.16 & \multicolumn{1}{r}{0.21} & 0.00029 & 0.00019 & 0.00031 & 0.00143 \\ 
1.1 & -0.4 & 0.13 & ... & 0.00037 & 0.00024 & 0.00036 & 0.00147 \\ 
1.3 & -0.4 & 0.07 & ... & 0.00045 & 0.00027 & 0.00045 & 0.00126 \\ 
1.5 & -0.4 & 0.10 & ... & 0.00054 & 0.00029 & 0.00053 & 0.00160 \\ 
0.9 & -0.2 & 0.20 & ... & 0.00037 & 0.00023 & 0.00036 & 0.00181 \\ 
1.1 & -0.2 & 0.11 & ... & 0.00048 & 0.00028 & 0.00045 & 0.00156 \\ 
1.3 & -0.2 & 0.09 & ... & 0.00059 & 0.00032 & 0.00055 & 0.00161 \\ 
1.5 & -0.2 & 0.10 & ... & 0.00069 & 0.00034 & 0.00065 & 0.00173 \\ 
1.1 & 0.0 & 0.14 & ... & 0.00059 & 0.00032 & 0.00054 & 0.00224 \\ 
1.3 & 0.0 & 0.05 & ... & 0.00072 & 0.00037 & 0.00064 & 0.00210 \\ 
1.5 & 0.0 & 0.09 & ... & 0.00085 & 0.00039 & 0.00078 & 0.00225 \\ 
1.7 & 0.0 & 0.02 & ... & 0.00107 & 0.00042 & 0.00093 & 0.00275 \\ 
1.1 & 0.2 & 0.12 & ... & 0.00072 & 0.00037 & 0.00068 & 0.00226 \\ 
1.3 & 0.2 & 0.00 & ... & 0.00094 & 0.00043 & 0.00083 & 0.00317 \\ 
1.5 & 0.2 & 0.01 & ... & 0.00112 & 0.00047 & 0.00098 & 0.00267 \\ 
1.7 & 0.2 & 0.00 & ... & 0.00142 & 0.00051 & 0.00119 & 0.00318 \\ 
1.1 & 0.4 & 0.04 & ... & 0.00100 & 0.00045 & 0.00092 & 0.00289 \\ 
1.3 & 0.4 & 0.01 & ... & 0.00124 & 0.00052 & 0.00104 & 0.00319 \\ 
 \hline
\end{tabular}
\label{tab:obsdata}
\end{center}
\end{table*}
%\end{minipage}

Recently, modern spectroscopic surveys have begun collecting measurements of mixing diagnostics for large samples of stars whose masses are also well constrained, making it possible to compare %the observed trends of mixing 
{observed mixing trends}
to the predictions of the 
%variety of 
theoretical models discussed above in Section \ref{sec:parameterizations} by using one dimensional stellar evolution models to estimate the relevant fluid parameters of these stars given their masses and metallicities (Section \ref{sec:mesa_results}). 
%
We choose for this work to use the carbon-to-nitrogen ratios, {[C/N],} measured from the Apache Point Galactic Evolution Experiment \citep[APOGEE, ][]{Majewski2015,Majewski2017}. APOGEE is a Sloan Digital Sky Survey III and IV \citep{Blanton2017} project using the 2.5 meter Sloan Telescope \citep{Gunn2006} and the APOGEE spectrograph \citep{Wilson2019} to obtain medium resolution (R $\sim$ 22,500) spectra of large numbers of stars across the galaxy \citep{Zasowski2017, Beaton2021,Santana2021}. These spectra are homogeneously reduced and analyzed using the ASPCAP pipeline \citep{Nidever2015, Zamora2015, GarciaPerez2016} and the resulting stellar parameters are then calibrated using asteroseismic, cluster, and field data \citep{Holtzman2015,Holtzman2018, Jonsson2020}. We choose to use the APOGEE data because this calibration work has already been done and an asteroseismic overlap sample is already available to provide stars with precise and accurate masses, though %\sout{we acknowledge that} 
similar work could likely be done with, for example, the lithium abundances measured by the GALAH survey \citep{buder2019} or the \ctwelvecthirteen data estimated from the APOGEE data using the Brussels Automatic Code
for Characterizing High accUracy Spectra \citep[BACCHUS,][]{Masseron2016_BACCHUS} pipeline (C. Hayes, submitted). 

We also note that the evolution of [C/N] is in some ways simpler for these low-mass stars than other mixing diagnostics. Unlike lithium, its abundance at the surface does not change significantly during the main sequence due to the effects of rotational and other mixing processes \citep{Iben1967}. Its initial ratio seems to be somewhat metallicity dependent \citep{Shetrone2019}%, JackandMarcifSubmitted}
, with higher values at lower metallicity. As stars reach the first dredge up, there is a strong, rapid, mass--dependent change in the surface [C/N] ratio \citep{MasseronGilmore2015, Martig2016, Ness2016, Spoo2022}. The [C/N] ratio at the surface then remains constant until stars reach the red giant branch bump, after which there seems to be another rapid drop in the [C/N] ratio, particularly in stars of low metallicity \citep[e.g.][]{Gratton2000,Shetrone2019}; it is this drop that has been associated with thermohaline mixing. On the upper giant branch for stars of particularly low metallicities, there are some suggestions of Upper RGB extra mixing \citep{Shetrone2019} which is not well motivated theoretically, but it is separate from the processes we are discussing here.

To estimate the amount of extra mixing in these stars near the bump, which thermohaline models suggest should correlate with the %compositional Nusselt number (\Numu) and 
mixing coefficient $\Dth$ described above, %we follow the work of 
\citet{Shetrone2019} estimated the drop in [C/N] just above the red giant branch bump. Their work used $\alpha$-element enhanced, and therefore old and low-mass ($\sim$0.9 \msun), first ascent red giant branch stars and binned them in bins of 0.2 dex in metallicity. The location of the red giant branch bump was identified empirically as an overdensity of stars at a particular surface gravity in each bin. They then identified the %\sout{region of gravities}
$\log g$ regime around the red giant branch bump, and fit a hyperbolic tangent function to measure the location and size of the drops in the [C/N] ratio. For simplicity, we have reproduced their results in Table \ref{tab:obsdata}. 

We add to their analysis a sample of higher metallicity stars with asteroseismic masses from the APOGEE-Kepler overlap sample \citep[APOKASC,][]{Pinsonneault2014, Pinsonneault2018}. We do this because, according to our analysis in Section \ref{sec:mesa_results}, higher mass, higher metallicity stars probe larger values of the reduced density ratio, $r$. %Specifically,
We first bin the stars in mass (0.2 \msun) and  metallicity (0.2 dex). For consistency with \citet{Pinsonneault2018} and \citet{Shetrone2019}, we use the Data Release 14 \citet{DR14} carbon and nitrogen abundances. We note however that while the abundance scale seems to shift between releases, the rank ordering does not change very much \citep{Spoo2022}, which means that the conclusions of this analysis are not strongly affected by the choice of Data Release or seismic parameters.

Unlike in the \citet{Shetrone2019} analysis (e.g. their Figure 2), there is not a sufficient number of stars near the bump in each bin to detect and measure the extra mixing directly in the asteroseismic sample. Instead, we define a `pre-mixing' bin of stars between \logg\ of 3.4 and 2.8 dex whose oscillations have identified them as first ascent red giants \citep{Elsworth2019}, as well as a `post-mixing' bin of RGB stars with surface gravities between 2.3 and 1.0 dex. We then compute the average [C/N] of stars in each of the pre-mixing and post-mixing bins. If both bins had at least three stars, then the difference between the pre-mixing and post-mixing average [C/N] is plotted in Figure \ref{fig:obssquare}. Because of the calibration and choices in the analysis pipeline
%We then identify the range of surface gravities that represent stars around the red giant branch bump, adopting 1.5 $<$ \logg $<$ 2.9 dex for consistency with \citet{Shetrone2019}. Finally, we fit a hyperbolic tangent function to the [C/N] ratio as a function of surface gravity, reporting the midpoint of the transition, and the total change in [C/N] in Table \ref{tab:obs}.
%For this analysis, we use the most recent Data Release 17 \citep{DR17} measurements of carbon and nitrogen, combined with the most recent astroseismic results (M. Pinsonneault et al. 2022, in prep). Because of the challenges of calibration, as well as the changes to the analysis pipelines between data releases 
\citep[see e.g.][]{Holtzman2018,Jonsson2020, vsmith_apogee_dr16_2021}, the scale of the abundances, particularly for carbon and nitrogen, is somewhat uncertain.
%often change between data releases. While the rank ordering of stars of a particular surface gravity tends to be robust \citep{Martig2016,Ness2016}, 
There sometimes exist small trends with surface gravity and temperature that are not fully removed in the calibration process. This is notable in our measurement results here; in the highest mass, highest metallicity bins, we formally measure `unmixing' near the red giant branch bump, i.e. an increase rather than a decrease in the average [C/N] near the red giant branch bump, which is inconsistent both with theoretical expectations and with measurements from other sources. Following discussions with the APOGEE team (C. Hayes, private communication), we have decided to correct for these effects by correcting the bin with the most `unmixing' to have 0 mixing, and subtracting that change from all of the other bins under the assumption that the systematic measurement errors are consistent for stars of similar temperatures and gravities. Because we are most interested here in the trend in mixing amounts as a function of the stellar parameters, we do not expect this shift to alter the results of this analysis, but we emphasize that care should be taken by future users of this data. 

%\textbf{ if we end up having a bunch of authors on this, we should add Chris Hayes. }
%FIGURE omgcomp---------------------------------------------------
\begin{figure}[!tb]
\begin{center}
\includegraphics[width=9cm,clip=true, trim=0.5in 0in 0in 0in]{Mfeh5c_mixingCRGBdr14v6.eps}%[width=9cm, clip=true, trim=1in 1in 1in 1in]{./Figs/omgcomp.eps}
\caption{The difference in average [C/N] (box color) between stars significantly below the RGB bump and those significantly above the bump, as a function of stellar mass and metallicity. The gradient is consistent with previous work, with lower mass, lower metallicity stars having more extra mixing (purple), but there is clearly an unphysical `unmixing' trend (orange) that needs to be removed (see text). We highlight only bins with a sufficient number of stars both below and above the bump. \red{EA suggests labeling the rhs of the colorbar with the corrected values. JT says this sounds really annoying to figure out how to do, so i'm not doing it unless everyone says it's necessary} }
\label{fig:obssquare}
\end{center}
\end{figure}
%FIGURE omgcomp---------------------------------------------------




\section{Results} 
\label{sec:punchline}
%Punchline!

%make plot of delta [C/N] versus R0(1D), color code observed binned points by mass of the star

Given the measured amounts of mixing described in Section \ref{sec:obs} and the reduced density ratios $r$ computed for stars of various masses and metallicities described in Section \ref{sec:mesa_results}, it is now possible to compare the observations to the predictions of various thermohaline models from Section \ref{sec:formalism} and to assess whether the observed mixing is qualitatively consistent with any such theoretical prescription.
%that which would be driven by the thermohaline instability.
%
We show in Figure \ref{Fig:punchline} the corrected changes in [C/N] compared to the inferred reduced density ratios on axes analogous to those of Figure \ref{fig:parameterization_compare}, where the y axis represented the rate of mixing. % theoretical predictions were first introduced}. \textbf{
The four panels correspond to the four modeling configurations described in Section \ref{sec:mesa_experiment}. %\textcolor{red}{[Feel free to drop this, but: is there an easy way to remind the reader why the y axis on this figure is analogous to the y axis on figure 1? -AF]}

We first note that the observed trends are not strongly sensitive to the assumed 1D mixing model: mixing decreases with increasing reduced density ratio regardless of parameterization.
%
Besides this, there are similarities and differences between the data and the theoretical predictions, which are reproduced in Figure \ref{Fig:compare} for comparison. A key finding is that the observed mixing is strongly correlated with the fluid parameters as predicted; this is true for stars with different masses and metallicities but similar reduced density ratios. %\textbf{This is} consistent with theoretical predictions under the simplifying assumption that the mixing efficiency and magnetic field are not varying strongly with mass or metallicity. 
We observe a decrease in the amount of mixing as the density ratio increases, which is consistent with standard 1D prescriptions of thermohaline mixing but inconsistent with the prescription from \citet{harrington}, which was informed by magnetohydrodynamic simulations. %the prescription informed by 3D magnetohydrodynamic
%simulations that include strong magnetic fields \textcolor{red}{[I prefer ``" because we don't want to say ``it's not magnets", we just want to say ``it's not the trend HG19 said magnets produce" and I think that's an important distinction. If this change is accepted, I think the part between ``consistent with" and ``but inconsistent with" also needs some slight tweaking. -AF]}.
We also find that the range of average reduced density ratios probed by the observational data we have available here is much smaller than the range of density ratios simulated by and studied within the theoretical 3D fluid dynamics community \citep[e.g.][]{brown_etal_2013}, although we note that the full range of ratios do appear in each individual simulation (See Appendix \ref{app:movie}).% \textcolor{red}{[we should point out that: it's not that larger r's don't ever appear in our MESA models, since they do (see appendix C), it's just that the metric we're focusing on here appears to be probing (or something) these smaller r's. -AF]}. 
 %As such, we must be cognizant of the limitations of our analysis; for instance, the observations do not probe the high-$r$ regimes \sout{where magnetic fields are most important} \textbf{which are presumably important for determining how the thermohaline-unstable region propagates until it reaches the convective envelope [-AF]}. 


%note that the scatter is also important here- if the mixing actually depends on B field, and that depends randomly on the star (or on the stellar M/Z combo on average), then stars of the same R0 should have a range of mixi-ness. If R0 is the only parameter that matters, then mixing should strongly correlate with R0 with only minimal/ observational measurement scatter. 

%FIGURE solar---------------------------------------------------
\begin{figure*}[!tb]
\begin{center}
\includegraphics[width=\textwidth]{mixing_vs_r.pdf}%[width=9cm, clip=true, trim=1in 1in 1in 1in]{./Figs/omgcomp.eps}
\caption{Corrected measurements of the change in [C/N] near the red giant branch bump are compared to the reduced density ratio inferred from one-dimensional models using various thermohaline mixing prescriptions (Brown, Kippenhahn $\alpha_{\rm th}=0.1$, Kippenhahn $\alpha_{\rm th}=0.2$,Kippenhahn $\alpha_{\rm th}=0.700$). Observations are color coded by the metallicity bin of each data point. In general, there is a clear correlation between these parameters, suggesting that the observed mixing may indeed be related to %\textcolor{red}{[``related to" feels weak given how awesome of a result this is, is there a fair way to strengthen it a little? -AF]} 
the unstable mean molecular weight gradient. Mixing and the reduced density ratio $r$ are inversely correlated, which is consistent with hydrodynamic thermohaline prescriptions. %\sout{More mixing is observed when the fluid is more \textbf{unstable to the thermohaline instability [can this be said a different way?]}; this is consistent with standard theories of thermohaline mixing.} 
%We note that the observations do not probe high values of the reduced density ratios, which is where magnetic fields are most important. \textcolor{red}{[I am in favor of striking out this last sentence because the reader could take from this that we claim magnetic fields do not matter, when in fact this range of r values is still in the regime where HG19 says magnetic fields do matter -AF]}
\label{Fig:punchline}
}
\end{center}
\end{figure*}

\begin{figure*}[!tb]
\begin{center}
\includegraphics[width=\textwidth]{punchline.pdf}
\caption{\textbf{Left:} A reproduction of Figure \ref{fig:parameterization_compare} showing the predicted rate of mixing versus the reduced density ratio in various prescriptions of thermohaline mixing, including hydrodynamic (orange, green) and magnetohydrodynamic (purple) models. \textbf{Right:} The observed extra mixing near the red giant branch bump as a function of the reduced density ratio inferred from one dimensional stellar evolution models. While the conversion from the change in a mixing diagnostic to the fluid mixing rate is not trivial, and therefore we do not attempt it here, we note that the observed mixing amounts are strongly negatively correlated with $r$, with stars probing on average a relatively narrow range of the regime formally unstable to thermohaline mixing. }
\label{Fig:compare}
\end{center}
\end{figure*}




\section{Conclusions }
\label{sec:conclusions}
%
%
%
Thermohaline mixing has long been considered the most likely candidate for explaining the evolution of the surface chemistry of low--mass upper red giant branch stars. In this analysis, we have shown that:
%
% 
\begin{enumerate}
    \item prescriptions informed by three-dimensional simulations of the thermohaline instability suggest that mixing rates should strongly depend on the reduced density ratio, $r$ but the shape of the correlation varies from model to model; 
    %\textcolor{red}{[As currently stated, this reads as if we're taking credit for the stuff that we're actually just repeating from previous work. One way to fix this is to tweak this to say 
    %we have shown that prescriptions informed by 3D sims suggest qualitatively different dependencies for mixing rates vs r 
    %(this is true because HG19 never actually looked at the predicted r dependence from their prescription, so our paper is the first time anyone has done that) -AF]}
    
    \item one-dimensional stellar evolution models suggest that the average reduced density ratio, $r$, should vary as a function of mass and metallicity, with stars of lower masses and lower metallicities having smaller $r$;
    
    \item one-dimensional stellar evolution models suggest that the average reduced density ratio, $r$, is not strongly dependent on the assumed parameterized thermohaline mixing model (e.g. Brown vs Kippenhahn), and regardless of the choice of $\alpha_{\rm th}$ in the case of Kippenhahn; 
    
    \item one-dimensional stellar evolution models suggest that the average reduced density ratio, $r$, occupies only a small range ($10^{-4}<r<10^{-3}$) of the parameter space formally unstable to thermohaline mixing, $0<r<1$;
    
    \item observations suggest that the amount of mixing is strongly correlated with the reduced density ratio $r$ in a way that qualitatively agrees with predictions from three dimensional simulations;
    
    \item observations indicate that the mixing rate and reduced density ratio $r$ are inversely correlated, a finding that is consistent with 
    %\sout{hydrodynamic prescriptions but not \red{current implementations of the HG19 model} magnetohydrodynamic prescriptions (read: not only are they correlated, they're correlated in the way that hydro models predict!)} 
    currently available 1D prescriptions informed by 3D simulations, but not consistent with the prescription put forth by \citet{harrington}.
    %\textcolor{red}{[we need to tone this down, because there are known flaws with the HG19 prescription but we aren't poking any holes in the results of their simulations themselves, which unambiguously show that magnetic fields enhance mixing relative to hydrodynamic levels]}
    
    \end{enumerate}
    %

Most importantly, we find that our proposed framework for connecting observations of extra mixing in red giants to the fluid simulations of the thermohaline instability through the medium of one dimensional stellar evolution models is feasible and robust. 
%
It motivates more rigorous exploration into whether red giant branch extra mixing should be associated with the thermohaline instability, and it will facilitate the translation of observational information into theoretical simulations. As observational data sets improve, abundance measurements will only become more capable of constraining stellar interior mixing, the relevant fluid parameters, and potentially even the magnetic fields in these regions.

However, the present study is largely a proof of concept;
there is room for significant development in all aspects of this method. From an observational perspective, we have so far only considered the change in [C/N] as an observational mixing diagnostic, and only in a fairly restricted set of stars. Looking at higher or lower masses, using a variety of mixing diagnostics, or probing stars including globular clusters, binary stars, and dwarf galaxies---where the base composition and mixing history may be different---could all be illuminating expansions of this work. There is also the potential to look at the timing of the extra mixing, as well as mixing rate as a function of time. Further, it may be possible to investigate whether mixing can be connected to the sorts of asteroseismic diagnostics that probe the interior structure of a star, including its density profile \citep{KjeldsenBedding1995}, chemical discontinuities \citep{Verma2017}, internal rotation \citep{Gehan2018}, and magnetic fields \citep{Bugnet2021}. 

On the modeling side, we have explored a coarse but reasonable range of possible thermohaline conventions that could have impacted our conclusions, but this parameter exploration was certainly not exhaustive. It is worthwhile and necessary to test whether different 1D modeling assumptions impact the direction of this trend, particularly in the case of other mixing--related physical assumptions. Key variations in this regard include the choice of prescription for the Mixing Length Theory (MLT) formalism and value for the mixing length parameter, $\alpha_{\text{MLT}}$, the treatment of convective boundary mixing and convective overshoot, and the choice of atmospheric surface boundary conditions---all of which are well known to affect thermodynamic quantities in the regime we study here \citep{tayar2017, Joyce2018a, Joyce2018b, viani2018}. For similar thermodynamic reasons, it is also important to explore more extreme metallicity regimes, as the global metal content dictates the behavior of \gradmu.
%
Further, the need to introduce rotational mixing alongside thermohaline mixing in 1D stellar models to achieve the desired observational reproductions is likewise well established in the literature (c.f. \citealt{Charbonnel2010}), making the consideration of rotational effects an obvious candidate for future theoretical work.

Finally, while previous and ongoing work has endeavored to find a theoretical justification for why thermohaline mixing should be so efficient that it accounts for the entire amount of extra mixing observed after the RGB bump (by adding e.g.~magnetic fields or rotation), this work sets a new target for theoretical efforts: in order for thermohaline mixing to be the primary source of extra mixing, not only should there be a mixing prescription that agrees with 3D simulations and also reproduces the \textit{amount} of extra mixing, but such a prescription should also reproduce \textit{trends} in this extra mixing as a function of fundamental stellar parameters like mass and metallicity.
For instance, while \citet{harrington} demonstrate that magnetic fields of moderate strength can dramatically enhance the efficiency of thermohaline mixing---plausibly to levels that explain the full amount of extra mixing observed after the RGB bump---this work shows that their mixing prescription does not predict trends in mixing rate as a function of the density ratio that are consistent with observations.

In short, this paper has demonstrated the viability of comparing observed signatures of extra mixing on the red giant branch to the predictions of models informed directly by fluid simulations, and it has done so in the framework of the parameters defined by those simulations. We therefore anticipate and welcome future explorations that will produce better understanding of 
% (1)
whether extra mixing should indeed be associated with the thermohaline instability and 
% (2)
how observations of real stars---which probe a fluid regime well outside the regime we are currently able simulate--- can provide constraints on the physics of that instability.

%\sout{Cool suggestions for other projects that explore all the things boat boy has been worried about (\partyparrot).}




\begin{comment}
%FIGURE omgcomp---------------------------------------------------
\begin{figure}[!htb]
\begin{center}
\includegraphics[width=9cm,clip=true, trim=0.5in 0in 0in 0in]{./Figs/protversusloggmodelpmmPYboth.eps}%[width=9cm, clip=true, trim=1in 1in 1in 1in]{./Figs/omgcomp.eps}
\caption{The measured core rotation rates for the stars in our sample as a function of gravity compared to the predictions of our solid body model (lue) and our moerately differentially convection zone (pink), showing that these models provide limits on the allowable amount of radial differential rotation in the surface convection zone.}
%There seems to beevidence of core-envelope recoupling as the stars evolve on the secondary clump, although more measurements of evolved stars with slowly rotating surfaces would strengthen this conclusion.} %\textbf{mark a line for the selection effect?} }
\label{Fig:bothmodels}
\end{center}
\end{figure}
%FIGURE omgcomp---------------------------------------------------
\end{comment}

\begin{acknowledgements}
The first four authors of this manuscript contributed equally to this work. 
\begin{itemize}
\item[] A.~Fraser contributed the majority of text and was responsible for Secs.~2 and 3.%, and contributed other text.
%
\item[] M. Joyce wrote and tested the MESA modeling templates, including \texttt{inlists}, evolutionary and structural output, and \texttt{run\_star\_extras} functionality, wrote the data analysis and parameter extraction scripts in collaboration with E.H. Anders (Python), and contributed text.
%
\item[] E.H. Anders computed and analyzed stellar structure models, created Figs.~2 and 4-7, and contributed text.
%
\item[] J. Tayar was responsible for all analysis related to observations, constructed the initial manuscript template, and contributed text. 
%
\end{itemize}
Author order within this set was decided according to height in heels, descending.

The authors thank C. Hayes, A. Jermyn, and M. Pinsonneault for helpful discussions that contributed to this work and T. White for providing the script that translates between [Fe/H] and Z. The authors acknowledge the thermohaline working group at the KITP program ``Probes of transport in stars'' for fruitful discussion that led to the conception of this idea. The authors likewise thank the Space Telescope Science Institute for providing accommodation and infrastructure support during the week-long meeting during which the majority of this paper was written.
%
AF thanks Pascale Garaud for many discussions of thermohaline mixing which helped improve his understanding of the theory of that field. 
MJ thanks John Bourke for proofreading and helpful discussion of numerics.
EHA thanks MJ for her mentorship in teaching him how to run and interact with MESA models.
%
%
Support for this work was provided by NASA through the NASA Hubble Fellowship grant No.51424 awarded by the Space Telescope Science Institute, which is operated by the Association of Universities for Research in Astronomy, Inc., for NASA, under contract NAS5-26555. JT also acknowledges support from 80NSSC20K0056. AF acknowledges support from NSF Grant No.~AST-1908338. 
MJ acknowledges the Space Telescope Science Institute's Lasker Data Science Fellowship, the Kavli Institute for Theoretical Physics at UC Santa Barbara, and the MESA developers team.
EHA acknowledges the support of a CIERA Postdoctoral fellowship.
 The Center for Computational Astrophysics at the Flatiron Institute is
supported by the Simons Foundation.
 %
 Computations were conducted with support from the NASA High End Computing (HEC) Program through the NASA Advanced Supercomputing (NAS) Division at Ames Research Center on Pleiades with allocation GID s2276 which is provided through NASA HTMS grant 80NSSC20K1280. 
This research was supported in part by the National Science Foundation under Grant No. NSF PHY-1748958.


\end{acknowledgements}

\software{Astropy \citep{Astropy,Astropy_2018}, Matplotlib \citep{Matplotlib}, NumPy \citep{numpy}, SciPy \citep{2020SciPy-NMeth}}

\facilities{Du Pont (APOGEE), Sloan (APOGEE)} 

\bibliographystyle{aasjournal}
\bibliography{ms.bib, library.bib, library2.bib, thermohaline.bib, mesa.bib, Joyce_bibliography_4.12.22}


\appendix

\section{MESA}
\label{app:mesa}
The MESA EOS is a blend of the OPAL \citep{Rogers2002}, SCVH
\citep{Saumon1995}, FreeEOS \citep{Irwin2004}, HELM \citep{Timmes2000},
PC \citep{Potekhin2010}, and Skye \citep{Jermyn2021} EOSes.

Radiative opacities are primarily from OPAL \citep{Iglesias1993,
Iglesias1996}, with low-temperature data from \citet{Ferguson2005}
and the high-temperature, Compton-scattering dominated regime by
\citet{Poutanen2017}.  Electron conduction opacities are from
\citet{Cassisi2007}.

Nuclear reaction rates are from JINA REACLIB \citep{Cyburt2010}, NACRE \citep{Angulo1999} and
additional tabulated weak reaction rates \citet{Fuller1985, Oda1994,
Langanke2000}.  Screening is included via the prescription of \citet{Chugunov2007}.
Thermal neutrino loss rates are from \citet{Itoh1996}.


We create 1D stellar models and evolve them from the pre-main sequence until roughly the end of hydrogen shell burning.
We study stellar masses between 0.9 and 1.7 $M_{\odot}$ in steps of $0.2 M_{\odot}$.
We study metallicities [Fe/H] ranging from -1.4 to 0.4 in steps of 0.2 dex.
To convert from metallicity units to MESA input $Y$ and $Z$ units, we assume a linear helium enrichment law \citep[per e.g.,][sec 3.1]{choi2016} where we assume a big-bang $Y_p = 0.2485$ and $\Delta Y / \Delta Z = 1.3426$ according to table 1 of \citet{tayar_etal_2022}.
The algorithm we use to calculate $X$, $Y$, and $Z$ from these values is identical to the one used in \url{https://github.com/aarondotter/initial_xa_calculator}; we use the opacity tables of \citet{GrevesseSauval1998} and [$\alpha$/Fe].
The specific [Fe/H] to ($X$, $Y$, $Z$) conversions used here are shown in table~\ref{table:feh_to_z}.


\begin{deluxetable}{c c c c}
\tablehead{
\colhead{[Fe/H]} & \colhead{$X$} & \colhead{$Y$} & \colhead{$Z$}
}
\decimals
\startdata
      0.400 & 0.66214302 & 0.29971262 & 0.03814436 \\
      0.200 & 0.69253197 & 0.28229599 & 0.02517204 \\
      0.000 & 0.71318414 & 0.27045974 & 0.01635613 \\
     -0.200 & 0.72686070 & 0.26262137 & 0.01051793 \\
     -0.400 & 0.73576323 & 0.25751912 & 0.00671765 \\
     -0.600 & 0.74149343 & 0.25423501 & 0.00427157 \\
     -0.800 & 0.74515509 & 0.25213642 & 0.00270849 \\
     -1.000 & 0.74748410 & 0.25080161 & 0.00171429 \\
     -1.200 & 0.74896112 & 0.24995509 & 0.00108379 \\
     -1.400 & 0.74989606 & 0.24941926 & 0.00068468
\enddata
\begin{caption}
    Mappings between $[$Fe/H$]$ values and MESA input values of $(X, Y, Z)$.
    \label{table:feh_to_z}
\end{caption}
\end{deluxetable}



\section{Resolution testing}
\label{app:resolution_test}
We performed resolution tests for models with [Fe/H] $\in \{-1.2, -0.4, 0.4\}$ and $M \in \{0.9, 1.3, 1.7\}$ using the Brown thermohaline mixing prescription.
We studied a grid of \texttt{mesh\_delta\_coeff} and \texttt{time\_delta\_coeff} values which span from 0.1 to 1.0 over five log-space steps.
We measure $r$ in each of these models, and in Fig.~\ref{Fig:resolution_test} we plot the absolute value of the relative error between that $r$ value and the reference $r_{\rm ref}$ value reported for that case in Fig.~\ref{fig:mesa_r_spread}.
We calculate the relative error to be $1 - r/r_{\rm{ref}}$.

We find that small values of the mesh coefficient combined with large values of the time coefficient result in large errors.
This occurs because the front of the thermohaline zone, and sometimes the full thermohaline zone, becomes numerically unstable, and large oscillations in $R_0$ lead to large errors in the $r$ calculation.
Furthermore, we find that when the thermohaline front is not properly numerically resolved, it does not propagate upwards in mass coordinate and connect with the convective shell.


\begin{figure*}[!tb]
\begin{center}
\includegraphics[width=\textwidth]{resolution_test.pdf}
\caption{
    We plot the percent difference between the measured value of the reduced density ratio $r$ for a 5x5 resolution grid of MESA models with respect to the values reported in Fig.~\ref{fig:mesa_r_spread}.
    We study values of \texttt{mesh\_delta\_coeff} and \texttt{time\_delta\_coeff} each in five log-space steps between 0.1 and 1.
    We perform these resolution tests for a 3x3 grid of mass and metallicity with $M \in [0.9, 1.3, 1.7]$ and [Fe/H]$ \in [-1.2, -0.4, 0.4]$.
    Each colored grid in this figure represents a 5x5 resolution test at one value of $M$ and [Fe/H] as reported in the text label above the grid.
    The resolution of the grids of simulations presented in Fig.~\ref{fig:mesa_r_spread} are marked by black stars.
    Points whose percent difference from the reported values are colored in green, while points with larger percent difference are colored in red.
    }
\label{Fig:resolution_test}
\end{center}
\end{figure*}

\section{Movie of thermohaline front evolution}
\label{app:movie}
In Fig.~\ref{fig:movie}, we plot the stellar structure vs.~mass coordinate in the simulation which employs the Brown model and has $M = 1.1M_\odot$ and [Fe/H] $= -0.2$.
We limit the x-limits of the plot to the mass coordinate energy generation peak of the hydrogen burning shell on the left $m_{\rm max}$, and to $1.1 m_{\rm max}$ on the right.
An animated version of this figure is available online in the HTML version of the draft.

\begin{figure}[!tb]
\begin{center}
\includegraphics[width=5in]{R0_fig_003516.png}
\caption{
    Description
    }
\label{Fig:movie}
\end{center}
\end{figure}

\end{document}
